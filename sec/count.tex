% !TEX root = ../msimp.tex

\subsection{Counting generators}

We would like to stress that the number of non-degenerate multisimplices in $\sur_d(r)$ is much smaller than the number of non-degenerate simplices in $\BE_d(r)$.
For example,
\begin{align*}
	& P_\chi^{2,4}(x) = 24(1+6x+10x^2+5x^3) \\
	& P_\cE^{2,4}(x) = 24(1+23x+104x^2+196x^3+184x^4+86x^5+16x^6)
\end{align*}
and
\begin{align*}
	& P_\chi^{3,3}(x) = 6(1+3x+7x^2+9x^3+6x^4+x^5) \\
	& P_\cE^{3,3}(x) = 6(1+5x+25x^2+60x^3+70x^4+38x^5+8x^6 )
\end{align*}
where
\begin{align*}
	P_\chi^{d,r}(x) &= \sum_n \,\rank(\chi_d(r)_n) \cdot x^n, \\
	P_\cE^{d,r}(x)  &= \sum_n \,\rank(\cE_d(r)_n)  \cdot x^n.
\end{align*}
This makes the multisimplicial approach substantially more efficient than the simplicial
when performing computations. 
A calculation of obstruction to formality similar to that in
\cite{salvatore2020planarnonformality} took a full day with the simplicial model, and few seconds with the multisimplicial model.