\subsection{Counting generators}

Let us set $\chi_d(r):=\chains\sur_{d}(r)$. We stress that the number of generators in $\chi_d(r)$, corresponding to non-degenerate surjections, is much smaller than the corresponding number of generators in $\cE_d(r)$.
We give some examples.
Let us consider the generating polynomial functions counting the generators
$$PE_d^r(x) = \sum_i rank(\cE_d(r)_i) x^i $$ $$P\chi_d^r(x)=
\sum_i rank(\chi_d(r)_i) x^i$$
Then for example
\begin{align*}
	& PE_2^4(x)=24(1+23x+104x^2+196x^3+184x^4+86x^5+16x^6)\\
	& P\chi_2^4(x)=24(1+6x+10x^2+5x^3) \\
	& \\
	& PE_3^3(x) = 6(1+5x+25x^2+60x^3+70x^4+38x^5+8x^6 ) \\
	&  P\chi_3^3(x)= 6(1+3x+7x^2+9x^3+6x^4+x^5)
\end{align*}
Therefore the multisimplicial approach using $\chi$ is much more efficient than the simplicial
approach using $\cE$, when performing computations as in \cite{salvatore2020planarnonformality}.