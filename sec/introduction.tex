% !TEX root = ../msimplicial.tex


\section{Introduction} \label{s:introduction}

The normalized cochain complex of a simplicial set is equipped with the classical Alexander--Whitney cup product which has several explicit natural extensions to an $E_\infty$-algebra \cite{mcclure2003multivariable, berger2004combinatorial, medina2020prop1} a structure encoding commutativity and associativity up to coherent homotopies.
The importance of $E_\infty$-algebras is highlighted by a result of Mandell \cite{mandell2006homotopy_type} stating that finite type nilpotent spaces are weakly equivalent if and only if their singular cochains are quasi-isomorphic as $E_\infty$-algebras.
Our first goal is to define an explicit natural $E_\infty$-algebra structure on the normalized cochains of multisimplicial sets, which are generalizations of both simplicial and cubical sets and are useful for concrete computations since they can model spaces using fewer cells.

Multisimplicial sets are contravariant functors from products of the simplex category $\simplex$ to $\Set$.
Explicitly, for any positive integer $k$ the category $\mSet{k}$ of $k$-fold multisimplicial sets is the presheaf category $\Fun((\simplex^\op)^{\times k}, \Set)$.
There is a notion of geometric realization of a multisimplicial set, a CW complex with a cell $e_x$ for each non-degenerate multisimplex with a characteristic map from a product of geometric simplexes
\[
\gmsimplex{n_1}{n_k}.
\]
Said geometric realization induces a functor which, together with the singular $k$-fold multisimplicial functor, define a Quillen equivalence between a model category structure on $\mSet{k}$ and the usual category of topological spaces.
We are interested in algebraically model homotopy types, for which we consider the composition of the geometric realization and the functor of cellular chains.
This composition defines the functor of normalized chains $\chains$.
In \cref{ss:e-infty extension} we define a lift of its linear dual, the functor of normalized cochains, to the category of $E_\infty$-algebras.
As an application of this construction we describe in \cref{ss:cup coproducts} explicit generalizations of the cup-$i$ products defining Steenrod operations at all primes.

The restriction to the diagonal $\simplex^\op \subset (\simplex^\op)^{\times k}$ of any $k$-fold multisimplicial set $X$ defines a simplicial set $X^\diag$.
Quillen proved in \cite{Quillen}\todo{paolo:where is this? Could yo fill in the reference, please?} that there is a natural homeomorphism of realizations $\bars{X} \cong \bars{X^\diag}$.
Under this homeomorphism the cells of $\bars{X^\diag}$ arise from those of $\bars{X}$ through subdivision, a procedure described algebraically by the Eilenberg--Zilber quasi-isomorphism
\[
\EZ \colon \chains(X) \to \chains(X^\diag).
\]
The functor induced by the diagonal restriction has a right adjoint $\radj^{(k)}$ making the categories of $k$-fold multisimplicial and simplicial sets Quillen equivalent.
Furthermore, there is a natural inclusion
\[
\In \colon \chains(Y) \to \chains(\radj^{(k)} Y)
\]
which is also a quasi-isomorphism.
The linear dual of $\EZ$ is an algebra quasi-isomorphism, but it does not respect the $E_\infty$-algebra structure, whereas the linear dual of $\In$ is an $E_\infty$-algebra quasi-isomorphism.
Using the second statement we prove the following in \cref{ss:singular}.

\begin{corollary*}
	For any topological space $\fZ$, the linear map from its singular simplicial chains to its singular $k$-fold multisimplicial chains given by precomposing a continuous map $(\gsimplex^n \to \fZ)$ with the projection $(\gsimplex^n \times \gmsimplex{0}{0} \xra{\pi_1} \gsimplex^n)$ induces on cochains a natural quasi-isomorphism of $E_\infty$-algebras.
\end{corollary*}

\section{Introduction old}

The normalized cochain complex $N^*(K)$ of a simplicial set $K$ is equipped with the classical Alexander-Whitney cup product that makes into a differential graded algebra.
Our first goal is to extend this product to the case of multisimplicial sets.
Let us consider a $k$-fold simplicial set $X$, that is a contravariant functor from $(\Delta)^k$ to the category of sets.
The restriction to the diagonal $\Delta \subset (\Delta)^k$ defines a simplicial set $X^D$.
There is a notion of geometric realization $|X|$ of a $k$-fold simplicial set $X$, that is a CW complex with a cell $e_x$ for each non-degenerate multisimplex $x$, with a characteristic map from a product of simplexes $$\Delta_{i_1} \times \dots \times \Delta_{i_k} \to e_x$$ This extends the classical case where the characteristic map has a single simplex as domain.
Quillen proved in \cite{Quillen} that there is a natural homeomorphism of realizations $|X| \cong |X^D|$.
Under this homeomorphism the cells of $|X^D|$ arise from those of $|X|$ by subdividing $k$-fold products of simplexes into simplexes.
This procedure is described combinatorially by the Eilenberg-Zilber quasi-isomorphism
$$EZ:N_*(X) \to N_*(X^D)$$

%that induces a quasi-isomorphism on normalized chains
%$N_*(X) \to N_*(X^D)$ after quotienting out degenerate %chains.
% As in the classical simplicial case, the projection $C_*(X) \to N_*(X)$ onto normalized chains
%is a quasi-isomorphism.

\medskip

We prove in Theorem \ref{algebra} %quale?
that the cochain complex $N^*(X)$ is equipped with a differential graded algebra structure.
The product is the natural extension to the multisimplicial case of the cup product defined by the Alexander-Whitney formula, by
evaluating cochains on front and rear faces in all multisimplicial directions.
%formula?
We prove in section \ref{ultima} that the dual Eilenberg-Zilber map
$$EZ^*:N^*(X^D) \to N^*(X)$$ %provato dove?
 is a quasi-isomorphism of differential graded algebras, where the source is equipped with the classical cup product.
We extend the previous construction to an $E_\infty$ structure on multisimplicial cochains using the approach defined by the first author in \cite{anibal}.
In this case $EZ$ does not respect the $E_\infty$-structures, but there is a natural quasi-isomorphism in the opposite direction that does preserve them.
\paolo{anibal elaborates on Cartan-Serre?}

Our result is very useful for computations, since multisimplicial models of spaces have a significantly smaller number of non-degenerate cells then their simplicial models.
So $N^*(X)$ is much smaller than $N^*(X^D)$, but it contains the same information up to homotopy, allowing for example to calculate explicitly homology operations like Massey products, and the Steenrod algebra action.
As an example we consider a family of multisimplicial sets $Sur(k)$ defined by McClure and Smith, see \cite{MS}, modelling euclidean configuration spaces.
The proof by the third author of the non-formality of the cochain algebra of planar configuration spaces in \cite{formality} used the Barratt-Eccles simplicial model and the classical cup product.
Our new product on the multisimplicial McClure-Smith models makes the computation much simpler and faster, paving the way for an extension to higher dimensions.
%per esempio dire i numeri..

\medskip

Part of the results appeared in the B.Sc. thesis of the second author, University of Rome Tor Vergata (2019), written under the guidance of the third author.