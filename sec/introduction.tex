% !TEX root = ../msimplicial.tex

\section{Introduction} \label{s:introduction}

The normalized cochain complex of a simplicial set is equipped with the classical Alexander--Whitney cup product which has several explicit natural extensions to an $E_\infty$-algebra \cite{mcclure2003multivariable, berger2004combinatorial, medina2020prop1} a structure encoding commutativity and associativity up to coherent homotopies.
The importance of $E_\infty$-algebras is highlighted by a result of Mandell \cite{mandell2006homotopy_type} stating that finite type nilpotent spaces are weakly equivalent if and only if their singular cochains are quasi-isomorphic as $E_\infty$-algebras.
Our first goal is to define an explicit natural $E_\infty$-algebra structure on the normalized cochains of multisimplicial sets, which are generalizations of both simplicial and cubical sets; and are useful for concrete computations since they can model spaces using fewer cells.

Multisimplicial sets are contravariant functors from products of the simplex category $\simplex$ to $\Set$.
Explicitly, for any positive integer $k$ the category $\mSet{k}$ of $k$-fold multisimplicial sets is the presheaf category $\Fun((\simplex^\op)^{\times k}, \Set)$.
There is a notion of geometric realization of a multisimplicial set, a CW complex with a cell $e_x$ for each non-degenerate multisimplex with a characteristic map from a product of geometric simplexes
\[
\gmsimplex{n_1}{n_k}.
\]
Said geometric realization induces a functor which, together with the singular $k$-fold multisimplicial functor, define a Quillen equivalence between a model category structure on $\mSet{k}$ and the usual category of topological spaces.
We are interested in algebraically model homotopy types, for which we consider the composition of the geometric realization and the functor of cellular chains.
This composition defines the functor of normalized chains $\chains$.
In \cref{ss:e-infty extension} we define a lift of its linear dual, the functor of normalized cochains, to the category of $E_\infty$-algebras.
As an application of this construction we describe in \cref{ss:cup coproducts} explicit generalizations of the cup-$i$ products defining Steenrod operations at all primes.

The restriction to the diagonal $\simplex^\op \subset (\simplex^\op)^{\times k}$ of any $k$-fold multisimplicial set $X$ defines a simplicial set $X^\diag$.
Quillen proved in \cite{Quillen}\todo{paolo:where is this? Could yo fill in the reference, please?} that there is a natural homeomorphism of realizations $\bars{X} \cong \bars{X^\diag}$.
Under this homeomorphism the cells of $\bars{X^\diag}$ arise from those of $\bars{X}$ through subdivision, a procedure described algebraically by the Eilenberg--Zilber quasi-isomorphism
\[
\EZ \colon \chains(X) \to \chains(X^\diag).
\]
The functor induced by the diagonal restriction has a right adjoint $\radj^{(k)}$ making the categories of $k$-fold multisimplicial and simplicial sets Quillen equivalent.
Furthermore, there is a natural inclusion
\[
\In \colon \chains(Y) \to \chains(\radj^{(k)} Y)
\]
which is also a quasi-isomorphism.
The linear dual of $\EZ$ is an algebra quasi-isomorphism, but it does not respect the $E_\infty$-algebra structure, whereas, as proven in \cref{ss:inclusion}, the linear dual of $\In$ is an $E_\infty$-algebra quasi-isomorphism.
We use the latter to to prove in \cref{ss:singular} that, for any topological space $\fZ$, the linear map from its singular simplicial chains to its singular $k$-fold multisimplicial chains given by precomposing a continuous map $(\gsimplex^n \to \fZ)$ with the projection $(\gsimplex^n \times \gmsimplex{0}{0} \xra{\pi_1} \gsimplex^n)$ induces on cochains a natural quasi-isomorphism of $E_\infty$-algebras.

%Using the second statement we prove the following in \cref{ss:singular}.

%\begin{corollary*}
%	For any topological space $\fZ$, the linear map from its singular simplicial chains to its singular $k$-fold multisimplicial chains given by precomposing a continuous map $(\gsimplex^n \to \fZ)$ with the projection $(\gsimplex^n \times \gmsimplex{0}{0} \xra{\pi_1} \gsimplex^n)$ induces on cochains a natural quasi-isomorphism of $E_\infty$-algebras.
%\end{corollary*}

Our motivation for studying multisimplicial algebraic models of homotopy types comes from the observation that, in some cases of interest, such models have a smaller number of generators compared to simplicial or cubical ones.
We illustrate this point using McClure--Smith's \cite{mcclure2003multivariable} equivariant algebraic model of the configuration space $Con(r,d)$ of $r$ ordered distinct points in $\R^d$, which we recognize as resulting from applying the functor of normalized chains to an $r$-fold multisimplicial set $Sur(r,d)$.
McClure--Smith also considered an operad structure on these, but we do not focus on it.
With respect to the $E_\infty$-algebra structures described above (\cref{ss:e-infty extension}), we construct in \cref{ss:multisimplicial model} an explicit zig-zag of $E_\infty$-algebra quasi-isomorphism from the normalized cochains of $Sur(r,d)$ to the singular (simplicial) cochains of $Con(r,d)$.

There is also a simplicial model $W(r,d)$ studied by Berger--Fresse \cite{berger2004combinatorial}.
We construct a factorization of the table completion quasi-isomorphism of these authors as
\[
\mathrm{TC} \colon
\chains(Sur(r,d))
\xra{\EZ}
\chains(Sur(r,d)^\diag)
\xra{\chains(\mathrm{tc})}
\chains(W(r,d)),
\]
where the second map is induced from an explicit simplicial map.
Since $\EZ$ is only a coalgebra map, this direct map does not preserve the $E_\infty$-structure, but we construct a zig-zag of quasi-isomorphism of $E_\infty$-coalgebras between them.

The proof by the third author of the non-formality of the cochain algebra of planar configuration spaces in \cite{formality} used part of the $E_\infty$-algebra structure on $\cochains(W(r,d))$.
This chain complex has many more generators than $\cochains(Sur(r,d))$.
Our $E_\infty$-algebra structure on $\cochains(Sur(r,d))$ makes the computation much simpler and faster, paving the way for an extension to higher dimensions.
%per esempio dire i numeri..

\medskip

Part of the results appeared in the B.Sc. thesis of the second author, University of Rome Tor Vergata (2019), written under the guidance of the third author.