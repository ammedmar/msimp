% !TEX root = ../msimplicial.tex

\section{Introduction}\todo{@everyone: to be updated.} \label{s:introduction}

The normalized cochain complex $N^*(K)$ of a simplicial set $K$ is equipped with the classical Alexander-Whitney cup product that makes into a differential graded algebra.
Our first goal is to extend this product to the case of multisimplicial sets.  Let us consider a $k$-fold simplicial set $X$, that is a contravariant functor from $(\Delta)^k$ to the category of sets. The restriction to the diagonal  $\Delta \subset (\Delta)^k$ defines a simplicial set $X^D$. There is a notion
of geometric realization $|X|$ of a $k$-fold simplicial set $X$, that is a CW complex with a cell $e_x$ for each non-degenerate multisimplex $x$, with a characteristic map from a product of simplexes  $$\Delta_{i_1} \times \dots \times \Delta_{i_k} \to e_x$$ This extends the classical case where the characteristic map has a single simplex as domain.
Quillen proved in
\cite{Quillen} that there is  a natural homeomorphism of realizations $|X| \cong |X^D|$. Under this homeomorphism the cells of $|X^D|$ arise from those
of $|X|$ by subdividing $k$-fold products of simplexes into simplexes. This procedure is described combinatorially by the Eilenberg-Zilber quasi-isomorphism $$EZ:N_*(X) \to N_*(X^D)$$

%that induces a quasi-isomorphism on normalized chains
%$N_*(X) \to N_*(X^D)$ after quotienting out degenerate %chains.
% As in the classical simplicial case, the projection $C_*(X) \to N_*(X)$ onto normalized chains
%is a quasi-isomorphism.

\medskip

 We prove in Theorem \ref{algebra}  %quale?
 that the cochain complex $N^*(X)$ is equipped with a differential graded algebra structure.
 The product is the natural extension to the multisimplicial case of the cup product  defined by the Alexander-Whitney formula, by
  evaluating cochains on front and rear faces in all multisimplicial directions. %formula?
  We prove in section \ref{ultima}
   that the dual Eilenberg-Zilber map
 $$EZ^*:N^*(X^D) \to N^*(X)$$ %provato dove?
  is a quasi-isomorphism of differential graded algebras, where the source is equipped with the classical cup product.
We extend the previous construction to an $E_\infty$ structure on multisimplicial cochains using the approach defined by the first author in \cite{anibal}.
In this case $EZ$ does not respect the $E_\infty$-structures, but there is a natural quasi-isomorphism in the opposite direction that does preserve them.
\paolo{anibal elaborates on Cartan-Serre?}

 Our result is very useful for computations, since multisimplicial models of spaces have a significantly smaller number of non-degenerate cells then their simplicial models.
So $N^*(X)$ is much smaller than $N^*(X^D)$, but it contains the same information up to homotopy,
allowing for example
to calculate explicitly homology operations like Massey products, and the Steenrod algebra action.
As an example we consider a family of multisimplicial sets $Sur(k)$ defined  by McClure and Smith, see \cite{MS},  modelling euclidean configuration spaces.
The proof by the third author of the non-formality of the cochain algebra of planar configuration spaces in \cite{formality}  used the Barratt-Eccles simplicial model and the classical cup product.
Our new product on the multisimplicial McClure-Smith models makes the computation much simpler and faster, paving the way for an extension to higher dimensions.
%per esempio dire i numeri..

\medskip

Part of the results appeared in the B.Sc. thesis of the second author, University of Rome Tor Vergata (2019), written under the guidance of the third author.