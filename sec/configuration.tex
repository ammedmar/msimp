% !TEX root = ../msimp.tex

\section{\pdfEinfty-models of configuration spaces}

We are interested in modeling algebraically the equivariant homotopy type of the configuration space of $r$ labeled points in Euclidean $d$-dimensional space.
Multisimplicial sets can be used to provide an explicit chain complex model with a small number of generators.
By Mandell's theorem \cite{mandell2006homotopy_type} and the $E_\infty$-structure introduced in \cref{ss:e-infty extension} this model retains all homotopical information.

In the first subsection we recall a method due to Berger detecting spaces homotopy equivalent to euclidean configuration spaces by means of a filtration indexed by a {\em complete graph poset}.    
In the second subsection we construct the multisimplicial model and show that is equipped with such a filtration.
In the third subsection we recall the construction of the simplicial Barratt-Eccles model and show that is  equipped with a similar filtration.
In the fourth subsection we relate the multisimplicial and simplicial chain models by an explicit map. In the last subsection we give some examples of sizes of the two models, showing that the multisimplicial is smaller.
\subsection{Recognition of configuration spaces}\label{ss:recognition}



Let $\con{r}(\R^{d})$ denote the configuration space of $r$-tuples of pairwise disjoint vectors in $\R^{d}$.
This space is equipped with a free action of the symmetric group $\Sigma_r$ of permutations of $\{1,\dots,r\}$ swapping elements of a $r$-tuple.

%In this section we refine the filtrations of McClure-Smith and Barratt--Eccles in order to show that $tc$ induces equivariant weak equivalence on the filtation terms.

\begin{definition}[Complete graphs]
	A \textit{complete graph} on $r$ vertices is a pair $(\mu,\sigma)$, where $\mu$ is a collection of non-negative integers $\mu_{ij}\in\mathbb{N}$, $1 \leq i < j \leq r$
	and $\sigma$ is an ordering of
	$\{1,\dots,r\}$. Graphically $(\mu,\sigma)$ is a simple directed graph in which every couple of vertices $i,j$ is labeled with $\mu_{ij}$ and oriented consistently with the ordering.
	Let us denote the set of complete graphs with $r$ vertices as $\mathcal{CG}(r)$ and with $\mathcal{CG}_{n}(r)$ the collection of those graphs with $\max_{i<j}(\mu_{ij})< n$. We write $\sigma_{ij}$ for the restriction of the ordering $\sigma$ to the set $\{i,j\}$.
\end{definition}
	For example, the following figure represents a complete graph on 4 vertices with ordering $\sigma=(1432)$ and $\mu=(\mu_{12},\mu_{13},\mu_{14},\mu_{23},\mu_{24},\mu_{34})=(2,1,3,1,2,4)$.
	\begin{equation*}
		\begin{tikzcd}
			\ & 2 & \  \\
			1 \arrow[ur,"2"] \arrow[rr] \arrow[dr,"3"']& \arrow[r,"1"] \arrow[u,"2"] & 3 \arrow[ul,"1"']  \\
			\ & 4 \arrow[ur,"4"'] \arrow[uu] & \
		\end{tikzcd}
	\end{equation*}
	

\begin{definition}[Complete graphs form a poset]
	$\mathcal{CG}(k)$ has a poset structure as follow:
	\begin{equation*}
		(\mu,\sigma)\le (\nu,\tau) \ \ \text{ if and only if } \ \ (\mu_{ij}<\nu_{ij}) \ \ \text{ or } \ \ (\mu_{ij},\sigma_{ij})= (\nu_{ij},\tau_{ij})
	\end{equation*}
	for each pair $\left\lbrace i,j\right\rbrace \subset\left\lbrace 1,\dots,k  \right\rbrace $. This poset structure descends naturally to $\mathcal{CG}_{n}(k)$.
\end{definition}
We recall a definition due to Berger \cite{berger1997confspacemodel}, compare also \cite{beuckelmann2021master}.
\begin{definition}
For a given poset $A$, a cellular $A$-decomposition of a topological space $X$ is a family of subspaces $X_a \subseteq X$ indexed by $a \in A$ such that
\begin{itemize}
    \item if $a \leq b$ then $X_a \subseteq X_b$;
\item $\bigcup_{a \in A} X_a = X$;
\item $X_a$ is contractible for each $a$;
\item the inclusion $\bigcup_{a<b} X_a \subset X_b$ is a closed cofibration.
\end{itemize}
\end{definition} 

\begin{proposition}
If a topological space $X$ admits a \textit{cellular $A$-decomposition} induced by a poset $A$, then there is a homotopy equivalence between $X$ and the geometric realization of $A$. 
\end{proposition}

The case of interest for us is the following.

\begin{theorem} (Berger,\cite[\S1.13]{berger1997confspacemodel})
If a space has a cellular $\mathcal{CG}_d(r)$-decomposition then it is homotopy equivalent to 
the configuration space 
$\con{r}(\R^{d})$.
\end{theorem}

We warn the reader that configuration spaces themselves do not admit a cellular decomposition with respect to graph complete posets, but with respect to some extended versions of these posets, as explained in \cite{beuckelmann2021master}.

%Unfortunately, Euclidean configuration spaces do not admit a cellular $\mathcal{CG}$-decomposition. However, it is still possible to give a combinatorial model for these spaces using Smith-filtration from \cite{smith1989filtration}, in the context of simplicial sets, or equivalently using Surjection-filtration, in the context of multisimplicial sets.




\subsection{Multisimplicial model} \label{ss:multisimplicial model}

We define for each positive integer $k$ a family of multisimplicial sets $\sur(k)$ equipped with a complete graph filtration
\[
\sur(k,0) \subset \sur(k,1) \subset \sur(k,2) \subset \dotsb
\]
which, by applying to it the functor of chains, recovers the algebraic models of configuration spaces developed by McClure--Smith \cite{mcclure2003multivariable}.

Also the geometric realizations $\mathcal{F}(k) = \bars{\sur(k)}$ appear in the work by McClure-Smith \cite{MS}, as explained in the appendix of \cite{Deligne}.
\todo{@paolo: this sentence appears cryptic to me. Can you say what you mean? Maybe in the previous sentence we can say not algebraic models but ``space'' level.}

Let $\sur(k)$ be the $k$-fold multisimplicial set that has as $(m_1,\dots,m_k)$-multisimplices
the surjective maps
\[
f \colon \set{1,\dots,m+k} \to \set{1,\dots,k}
\]
where $m = m_1+\dots+m_k$ satisfying that the cardinality of $f^{-1}(\ell)$ is $m_\ell$ for each $\ell \in \set{1,\dots,k}$.
We represent this multisimplex by the sequence $f(1) \dotsm f(m+k)$.
The face and degeneracy maps $d^\ell_j$ and $s^\ell_j$ act on it by respectively removing and doubling the $(j+1)^\th$ occurrence of $\ell$ in the sequence.
For example,
\begin{align*}
	d^2_0(12321)&=1321, & s^1_0(121)&=1121, \\
	d^2_1(12321)&=1231, & s^1_1(121)&=1211.
\end{align*}
Degenerate multisimplices are exactly the sequences containing two equal adjacent terms.
A surjection belongs to $\sur(k,d)$ if any ordered subsequence with two distinct alternating values has length at most $d+1$.
For example, $x = 12132 \in \sur_3(3)$ but $x \notin \sur_2(3)$ because the subsequence $1212$ has length 4.
This is a complete graph filtration since...
\todo{@paolo: verify the conditions of complete graph filtration to get the (now commented out) proposition below}
The symmetric group $\Sigma_k$ acts on $\sur(k)$ by post-composition and the action preserves the filtration.

%\begin{proposition} \label{sur-real}
%	There is a $\Sigma_k$-equivariant homotopy equivalence
%	$\bars{\sur_d(k)} \simeq F_k(\R^d)$
%\end{proposition}
%
%\begin{proof}
%	The geometric realization $\bars{\sur_d(k)}$ appears in section 13 of \cite{MS} where it is denoted $Z_0^k$.
%	McClure and Smith define for each $d$ a topological operads $\mathcal{D}_d$ such that $\mathcal{D}_d(k)$ is $\Sigma_k$-homeomorphic to $\bars{\sur_d(k)} \times Tot(\Delta^*)$, where $Tot(\Delta^*)$ is contractible and equipped with trivial $\Sigma_k$-action.
%	This is explained in the appendix of \cite{cyclic} where the realization is constructed as a CW complex and denoted $\mathcal{F}_d(k)$.
%	McClure and Smith proceed to constructing a zig-zag of weak equivalences of operads between $\mathcal{D}_d$ and the little $d$-cubes operad
%	$\mathcal{C}_d$, that in particular gives a levelwise $\Sigma_k$-equivariant equivalence $\mathcal{D}_d(k) \simeq \mathcal{C}_d(k)$ for each $k$. It is well known that $\mathcal{C}_d(k)$ is $\Sigma_k$-equivariantly homotopy equivalent to $F_k(\R^d)$ and this concludes the proof.
%\end{proof}


%The functor of chains applied to these multisimplicial sets induces chain complex $\chi(k,d) \defeq \chains(\sur(k,d))$ with a filtration
%\[
%\chi(k,0) \subset \chi(k,0) \subset \chi(k,2) \subset \dotsb,
%\]
%which were considered by McClure and Smith \cite{MS}, %actually other paper
%who constructed an operad structure on the collections of these complexes, the {\it surjection operad} $\chi$.
%gives a filtration of the surjection operad by suboperads
%$$\chi_0 \subset \chi_1 \subset \chi_2 \subset \dots \subset \chi$$

\begin{theorem*}
	For any configuration space $\con(r,d)$ the maps in the following zig-zag are $\sym_r$-equivariant quasi-isomorphisms of $E_\infty$-coalgebras, or more precisely of $\UM$-coalgebras:
	\[
	\schains\con(r,d) \leftarrow \schains \cgr(r,d) \to \schains\bars{\sur(r,d)} \to \schains^{(r)}\bars{\sur(r,d)} \leftarrow \chains(\sur(r,d))
	\]

\end{theorem*}

\begin{proposition} \label{sur-model}
	$\chi_d(k)$ and $\rS{F_k(\R^d)}$ are $\Sigma_k$-equivariantly quasi-isomorphic $E_\infty$-coalgebras
\end{proposition}

\begin{proof}
	The $\Sigma_k$-equivariant homotopy equivalence of proposition \ref{sur-real} induces a $\Sigma_k$-equivariant quasi-isomorphism of $E_{\infty}$-coalgebras
	$$S^{(k)}(\bars{\sur_d(k)}) \simeq S^{(k)}(F_k(\R^d))$$.
	The weak equivalence %say that it is Quillen adjunction
	$\sur_d(k) \to Sing^{(k)}(\bars{\sur_d(k)})$ induces on the chain level a $\Sigma_k$-equivariant quasi-isomorphism of
	$E_\infty$-coalgebras
	between $\chi_d(k)$
	and $S^{(k)}(\bars{\sur_d(k)})$.
	Finally the map of lemma \ref{} gives a $\Sigma_k$-equivariant quasi-isomorphism $S^{(1)}(F_k(\R^d)) \to S^{(k)}(F_k(\R^d))$ of $E_\infty$-coalgebras.
\end{proof}

% !TEX root = ../msimp.tex

\subsection{Simplicial model}\label{ss:simplicial model}

We recall the Barratt--Eccles simplicial set $\BE(r)$ defined for each $r\in\N$ and equipped with a $\CG(r)$-filtration
\[
\BE_{1}(r) \subset \BE_{2}(r) \subset \dotsb.
\]
which, by applying the functor of chains to it, recovers an algebraic models
\[
\cE_{1}(r) \subset \cE_{2}(r) \subset \dotsb
\]
of configuration spaces.

The $n$-simplices of $\BE(r)$ are tuples of $n+1$ elements of the symmetric group $\sym_r$.
Its face and degeneracy maps are defined by removing and doubling elements respectively.
There is an operad structure on these simplicial sets, but we do not consider it here.

For $i<j$ and $\sigma$ in $\sym_r$ let $\sigma_{ij}$ be the associated permutation in $\sym_2$.
%An element $(w_1,\dots w_n) \in \BE(r)$ and $i<j$, let $w_{ij} = (w_{1,ij}, \dots, w_{d,ij})$ with $w_{\ell,ij}$ the subsequence of $w_\ell$ obtained by omitting all the occurrences of elements different from $i$ and $j$.
%	For example, let $(123,231,312) \in (\BE_3)_{3}$, we have $(123,231,312)_{12}=(12,21,12)$, $(123,231,312)_{23}=(23,23,32)$ and $(123,231,312)_{13}=(13,31,31 )$.
The element $(w_1,\dots w_n)$ in $\BE(r)_{n}$ belongs to $\BE_{(\mu,\sigma)}(r)$ if for all $i<j$ the cardinality of $\set{w_{\ell,ij} \neq w_{\ell+1,ij}}$ is either less than $\mu_{ij}$ or equal to it and $(w_1)_{ij} = \sigma_{ij}$.
Therefore, $(w_1,\dots w_n)$ is in $\BE_{d}(r)$ iff the cardinality associated to each $i<j$ is less than $d$.

We notice that the action of $\sym_r$ on $\BE(r)$ preserves this $\CG(r)$-filtration.
Please consult \cite{smith1989filtration,kashiwabara1993confcomplex,berger1997confspacemodel} for more details. In terms of cellular $\CG(r)$-decompositions, applying realization functor to the above $\CG(r)$-filtration results in a cellular $\CG_d(r)$-decomposition of each $\bars{\BE_{d}(r)}$, then we obtain by proposition \ref{p:berger} the homotopy equivalence
$$\conf{r}{d}\cong \bars{\BE_{d}(r)}.$$

Applying the functor of singular chains gives
$\schains\conf{r}{d} \to \schains\bars{\BE_{d}(r)}$
a quasi-isomorphisms of $\UM$-coalgebras and using the unit we get a zig-zag of quasi-isomorphisms of $\UM$-coalgebras
\begin{equation}\label{eq:simplicial zig-zag}
	\schains\conf{r}{d} \cong
	\schains\bars{\BE_d(r)} \leftarrow
	\chains \BE_d(r)= \cE_d(r).
\end{equation}
Combining this with the zig-zag constructed in the previous subsection yields the following.

\begin{theorem}
	The chains on the simplicial and multisimplicial models of a configuration space are related by an explicit zig-zag of quasi-isomorphism of $E_\infty$-coalgebras.
\end{theorem}

\subsection{Counting generators}

Let us set $\chi_d(r):=\chains\sur_{d}(r)$. We stress that the number of generators in $\chi_d(r)$, corresponding to non-degenerate surjections, is much smaller than the corresponding number of generators in $\cE_d(r)$.
We give some examples.
Let us consider the generating polynomial functions counting the generators
$$PE_d^r(x) = \sum_i rank(\cE_d(r)_i) x^i $$ $$P\chi_d^r(x)=
\sum_i rank(\chi_d(r)_i) x^i$$
Then for example
\begin{align*}
	& PE_2^4(x)=24(1+23x+104x^2+196x^3+184x^4+86x^5+16x^6)\\
	& P\chi_2^4(x)=24(1+6x+10x^2+5x^3) \\
	& \\
	& PE_3^3(x) = 6(1+5x+25x^2+60x^3+70x^4+38x^5+8x^6 ) \\
	&  P\chi_3^3(x)= 6(1+3x+7x^2+9x^3+6x^4+x^5)
\end{align*}
Therefore the multisimplicial approach using $\chi$ is much more efficient than the simplicial
approach using $\cE$, when performing computations as in \cite{salvatore2020planarnonformality}.

