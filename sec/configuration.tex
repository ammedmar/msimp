% !TEX root = ../msimp.tex

\section{\pdfEinfty-models of configuration spaces}

We are interested in modeling algebraically the equivariant homotopy type of the configuration space of $r$ labeled points in Euclidean $d$-dimensional space.
Multisimplicial sets can be used to provide an explicit chain complex model with a small number of generators.
By Mandell's theorem \cite{mandell2006homotopy_type} and the $E_\infty$-structure introduced in \cref{ss:e-infty extension} this model retains all homotopical information.

In the first subsection we recall a method due to Berger detecting spaces homotopy equivalent to euclidean configuration spaces by means of a filtration indexed by a {\em complete graph poset}.    
In the second subsection we construct the multisimplicial model and show that is equipped with such a filtration.
In the third subsection we recall the construction of the simplicial Barratt-Eccles model and show that is  equipped with a similar filtration.
In the fourth subsection we relate the multisimplicial and simplicial chain models by an explicit map. In the last subsection we give some examples of sizes of the two models, showing that the multisimplicial is smaller.
\subsection{Recognition of configuration spaces}\label{ss:recognition}

Let $\conf{r}{d}$ denote the configuration space of $r$-tuples of pairwise disjoint vectors in $\R^{d}$.
This space is equipped with a free action of the symmetric group $\sym_r$ of permutations of $\{1,\dots,r\}$ swapping elements of a $r$-tuple.

\begin{definition}
	A \textit{complete graph} on $r$ vertices is a pair $(\mu,\sigma)$ with $\mu$ a collection of non-negative integers $\mu_{ij}$ for all $1 \leq i < j \leq r$, and $\sigma$ is an ordering of $\{1,\dots,r\}$.
	We write $\sigma_{ij}$ for the restriction of the ordering $\sigma$ to the set $\{i,j\}$.
	Graphically $(\mu,\sigma)$ is a simple directed graph in the edge corresponding to $i<j$ directed according to $\sigma_{ij}$ and labeled by $\mu_{ij}$.
	Please consult \cref{f:complete graph} for a example.
	Let us denote the set of complete graphs with $r$ vertices by $\CG(r)$ equipped with the poset structure
	\begin{equation*}
		(\mu,\sigma)\le (\nu,\tau) \ \ \text{ if and only if } \ \
		(\mu_{ij}<\nu_{ij}) \ \ \text{ or } \ \
		(\mu_{ij},\sigma_{ij})= (\nu_{ij},\tau_{ij})
	\end{equation*}
	for each pair $i<j$.
	It is equipped with a filtration by subposets
	\[
	\CG_1(r) \subset \CG_2(r) \subset \dotsb
	\]
	where $\CG_n(r)$ consists of those graphs with $\max(\mu_{ij})< n$.
\end{definition}

\begin{figure}
	\centering
	\begin{equation*}
		\begin{tikzcd}
			\ & 2 & \ \\
			1 \arrow[ur,"2"] \arrow[rr] \arrow[dr,"3"']& \arrow[r,"1"] \arrow[u,"2"] & 3 \arrow[ul,"1"'] \\
			\ & 4 \arrow[ur,"4"'] \arrow[uu] & \
		\end{tikzcd}
	\end{equation*}
	\caption{A complete graph on 4 vertices with ordering $\sigma=(1432)$ and $\mu=(\mu_{12},\mu_{13},\mu_{14},\mu_{23},\mu_{24},\mu_{34})=(2,1,3,1,2,4)$.}
	\label{f:complete graph}
\end{figure}

\begin{definition}
	For a given poset $A$, a cellular $A$-decomposition of a topological space $X$ is a family of subspaces $X_a \subseteq X$ indexed by $a \in A$ such that:
	\begin{itemize}
		\item $a \leq b$ implies $X_a \subseteq X_b$;
		\item $\bigcup_{a \in A} X_a = X$;
		\item $X_a$ is contractible for each $a$;
		\item $\bigcup_{a<b} X_a \subset X_b$ is a closed cofibration.
	\end{itemize}
\end{definition}

\begin{proposition}
	If a topological space $X$ admits a \textit{cellular $A$-decomposition}, then there is a homotopy equivalence between $X$ and the geometric realization of $A$.
	\todo{@andrea,@paolo: We need a reference for this proposition. Is this Quillen's theorem A? I am also happy with saying something like: well know result of McCord ’66 and Quillen ’73 ... Also, I think this statement should be of the form: then the natural map ??? is a homotopy equivalence.}
\end{proposition}

The case of interest for us is the following.

\begin{proposition}[{\cite[\S1.13]{berger1997confspacemodel}}]\label{p:berger}
	If a space has a cellular $\CG_d(r)$-decomposition then it is homotopy equivalent to
	the configuration space
	$\conf{r}{d}$.
\end{proposition}

We warn the reader that configuration spaces themselves do not admit a cellular decomposition with respect to graph complete posets, but they do so with respect to some extended versions of these posets \cite{beuckelmann2021master}.
\begin{equation}\label{eq:extended complete graph map}
	...
\end{equation}
\todo{@andrea,@paolo: We therefore have natural homotopy equivalences ??? for each r and d.}
\subsection{Multisimplicial model} \label{ss:multisimplicial model}

We define for each positive integer $k$ a family of multisimplicial sets $\sur(k)$ equipped with a complete graph filtration
\[
\sur(k,0) \subset \sur(k,1) \subset \sur(k,2) \subset \dotsb
\]
which, by applying to it the functor of chains, recovers the algebraic models of configuration spaces developed by McClure--Smith \cite{mcclure2003multivariable}.

Also the geometric realizations $\mathcal{F}(k) = \bars{\sur(k)}$ appear in the work by McClure-Smith \cite{MS}, as explained in the appendix of \cite{Deligne}.
\todo{@paolo: this sentence appears cryptic to me. Can you say what you mean? Maybe in the previous sentence we can say not algebraic models but ``space'' level.}

Let $\sur(k)$ be the $k$-fold multisimplicial set that has as $(m_1,\dots,m_k)$-multisimplices
the surjective maps
\[
f \colon \set{1,\dots,m+k} \to \set{1,\dots,k}
\]
where $m = m_1+\dots+m_k$ satisfying that the cardinality of $f^{-1}(\ell)$ is $m_\ell$ for each $\ell \in \set{1,\dots,k}$.
We represent this multisimplex by the sequence $f(1) \dotsm f(m+k)$.
The face and degeneracy maps $d^\ell_j$ and $s^\ell_j$ act on it by respectively removing and doubling the $(j+1)^\th$ occurrence of $\ell$ in the sequence.
For example,
\begin{align*}
	d^2_0(12321)&=1321, & s^1_0(121)&=1121, \\
	d^2_1(12321)&=1231, & s^1_1(121)&=1211.
\end{align*}
Degenerate multisimplices are exactly the sequences containing two equal adjacent terms.
A surjection belongs to $\sur(k,d)$ if any ordered subsequence with two distinct alternating values has length at most $d+1$.
For example, $x = 12132 \in \sur_3(3)$ but $x \notin \sur_2(3)$ because the subsequence $1212$ has length 4.
This is a complete graph filtration since...
\todo{@paolo: verify the conditions of complete graph filtration to get the (now commented out) proposition below}
The symmetric group $\Sigma_k$ acts on $\sur(k)$ by post-composition and the action preserves the filtration.

%\begin{proposition} \label{sur-real}
%	There is a $\Sigma_k$-equivariant homotopy equivalence
%	$\bars{\sur_d(k)} \simeq F_k(\R^d)$
%\end{proposition}
%
%\begin{proof}
%	The geometric realization $\bars{\sur_d(k)}$ appears in section 13 of \cite{MS} where it is denoted $Z_0^k$.
%	McClure and Smith define for each $d$ a topological operads $\mathcal{D}_d$ such that $\mathcal{D}_d(k)$ is $\Sigma_k$-homeomorphic to $\bars{\sur_d(k)} \times Tot(\Delta^*)$, where $Tot(\Delta^*)$ is contractible and equipped with trivial $\Sigma_k$-action.
%	This is explained in the appendix of \cite{cyclic} where the realization is constructed as a CW complex and denoted $\mathcal{F}_d(k)$.
%	McClure and Smith proceed to constructing a zig-zag of weak equivalences of operads between $\mathcal{D}_d$ and the little $d$-cubes operad
%	$\mathcal{C}_d$, that in particular gives a levelwise $\Sigma_k$-equivariant equivalence $\mathcal{D}_d(k) \simeq \mathcal{C}_d(k)$ for each $k$. It is well known that $\mathcal{C}_d(k)$ is $\Sigma_k$-equivariantly homotopy equivalent to $F_k(\R^d)$ and this concludes the proof.
%\end{proof}


%The functor of chains applied to these multisimplicial sets induces chain complex $\chi(k,d) \defeq \chains(\sur(k,d))$ with a filtration
%\[
%\chi(k,0) \subset \chi(k,0) \subset \chi(k,2) \subset \dotsb,
%\]
%which were considered by McClure and Smith \cite{MS}, %actually other paper
%who constructed an operad structure on the collections of these complexes, the {\it surjection operad} $\chi$.
%gives a filtration of the surjection operad by suboperads
%$$\chi_0 \subset \chi_1 \subset \chi_2 \subset \dots \subset \chi$$

\begin{theorem*}
	For any configuration space $\con(r,d)$ the maps in the following zig-zag are $\sym_r$-equivariant quasi-isomorphisms of $E_\infty$-coalgebras, or more precisely of $\UM$-coalgebras:
	\[
	\schains\con(r,d) \leftarrow \schains \cgr(r,d) \to \schains\bars{\sur(r,d)} \to \schains^{(r)}\bars{\sur(r,d)} \leftarrow \chains(\sur(r,d))
	\]

\end{theorem*}

\begin{proposition} \label{sur-model}
	$\chi_d(k)$ and $\rS{F_k(\R^d)}$ are $\Sigma_k$-equivariantly quasi-isomorphic $E_\infty$-coalgebras
\end{proposition}

\begin{proof}
	The $\Sigma_k$-equivariant homotopy equivalence of proposition \ref{sur-real} induces a $\Sigma_k$-equivariant quasi-isomorphism of $E_{\infty}$-coalgebras
	$$S^{(k)}(\bars{\sur_d(k)}) \simeq S^{(k)}(F_k(\R^d))$$.
	The weak equivalence %say that it is Quillen adjunction
	$\sur_d(k) \to Sing^{(k)}(\bars{\sur_d(k)})$ induces on the chain level a $\Sigma_k$-equivariant quasi-isomorphism of
	$E_\infty$-coalgebras
	between $\chi_d(k)$
	and $S^{(k)}(\bars{\sur_d(k)})$.
	Finally the map of lemma \ref{} gives a $\Sigma_k$-equivariant quasi-isomorphism $S^{(1)}(F_k(\R^d)) \to S^{(k)}(F_k(\R^d))$ of $E_\infty$-coalgebras.
\end{proof}

% !TEX root = ../msimp.tex

\subsection{Simplicial model}\label{ss:simplicial model}

We recall the Barratt--Eccles simplicial set $\BE(r)$ defined for each $r\in\N$ that is equipped with a $\CG(r)$-filtration.
Applying the functor of chains to the nested sequence
\[
\BE_{1}(r) \subset \BE_{2}(r) \subset \dotsb.
\]
will provide the algebraic models
\[
\cE_{1}(r) \subset \cE_{2}(r) \subset \dotsb
\]
of configuration spaces studied by Berger and Fresse in \cite{berger2004combinatorial}.

The $n$-simplices of $\BE(r)$ are tuples of $n+1$ elements of the symmetric group $\sym_r$.
Its face and degeneracy maps are defined by removing and doubling elements respectively.
There is an operad structure on these simplicial sets, but we do not consider it here.

Next we recall a $\CG(r)$-filtration on $\BE(r)$.
For $i<j$ and $\sigma$ in $\sym_r$ let $\sigma_{ij}$ be the associated permutation in $\sym_2$.
Given $(\mu,\sigma) \in \CG(r)$ then an
element $w = (w_0,\dots w_n) \in \BE(r)_n$,
$w \in \BE(r)_{(\mu,\sigma)}$ if for each $i<j$, the cardinality of $\set{\ell \mid (w_{\ell})_{ij} \neq (w_{\ell+1})_{ij}}$ is either less than $\mu_{ij}$ or equal to it and $(w_0)_{ij} = \sigma_{ij}$.

In particular $w$ has complexity $d$ or less if for each $i<j$
%the cardinality of $\set{\ell=1,\dots,n-1 \mid w_{\ell-1,ij} \neq w_{\ell,ij}}$ is less than $d$, i.e., 
the non-degenerate dimension of $w_{ij}=((w_0)_{ij},\dots,(w_n)_{ij})$ in $\BE(2)$ is $d$ or less for all $i<j$.
We notice that the action of $\sym_r$ on $\BE(r)$ preserves the nested sequence
$$\BE_1(r) \subset \BE_2(r) \subset \dots$$
For a proof that this is a $\CG(r)$-filtration we refer to Example 2.8 in \cite{berger1997confspacemodel}.
Please consult \cite{smith1989filtration,kashiwabara1993confcomplex,berger1997confspacemodel} for more details.
 
Applying the functor of singular chains to the zig-zag of \cref{p:zig-zag conf} produces a zig-zag of equivariant quasi-isomorphisms of $\UM$-coalgebras connecting $\schains\bars{\BE_d(r)}$ and $\schains\conf{r}{d}$.
Using the unit of the Quillen equivalence extends this zig-zag to one relating $\chains\BE_d(r)$ and $\schains\conf{r}{d}$, which can be combined with the zig-zag constructed in the previous subsection.
As announced in the introduction, this construction relates the chains on the multisimplicial model of configuration space and those the simplicial model via an explicit zig-zag of equivariant quasi-isomorphisms of $E_\infty$-coalgebras.



\subsection{Counting generators}

Let us set $\chi_d(r):=\chains\sur_{d}(r)$. We stress that the number of generators in $\chi_d(r)$, corresponding to non-degenerate surjections, is much smaller than the corresponding number of generators in $\cE_d(r)$.
We give some examples.
Let us consider the generating polynomial functions counting the generators
$$PE_d^r(x) = \sum_i rank(\cE_d(r)_i) x^i $$ $$P\chi_d^r(x)=
\sum_i rank(\chi_d(r)_i) x^i$$
Then for example
\begin{align*}
	& PE_2^4(x)=24(1+23x+104x^2+196x^3+184x^4+86x^5+16x^6)\\
	& P\chi_2^4(x)=24(1+6x+10x^2+5x^3) \\
	& \\
	& PE_3^3(x) = 6(1+5x+25x^2+60x^3+70x^4+38x^5+8x^6 ) \\
	&  P\chi_3^3(x)= 6(1+3x+7x^2+9x^3+6x^4+x^5)
\end{align*}
Therefore the multisimplicial approach using $\chi$ is much more efficient than the simplicial
approach using $\cE$, when performing computations as in \cite{salvatore2020planarnonformality}.

