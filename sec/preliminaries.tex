
\section{Conventions and preliminaries} \label{s:preliminaries}

\subsection{Chain complexes }

Throughout this article $\k$ denotes a commutative and unital ring and we work over its associated closed symmetric monoidal category of homological graded chain complexes $(\Ch, \otimes, \k)$, where we denote by $\Hom(C, C^\prime)$ the chain complex of $\k$-linear maps between the chain complexes $C$ and $C^\prime$, and refer to the functor $\Hom(-, \k)$ as \textit{linear duality}.

\subsection{Category theory}

Given categories $\mathsf{B}$ and $\C$ we denote their associated \textit{functor category} by $\Fun(\mathsf{B}, \C)$.
Recall that a category is said to be \textit{small} if its objects and morphisms form sets.
We denote the Cartesian monoidal category of small categories by $(\Cat, \times, \mathbb{1})$.

A category is said to be \textit{cocomplete} if any functor to it from a small category has a colimit.
If $\mathsf{A}$ is small and $\mathsf{C}$ cocomplete, then the \textit{(left) Kan extension of $g$ along $f$} exists for any pair of functors $f$ and $g$ in the diagram below, and it is the initial object in $\Fun(\mathsf{B}, \mathsf{C})$ making
\begin{equation*}
\begin{tikzcd}[column sep=normal, row sep=normal]
\mathsf{A} \arrow[d, "f"'] \arrow[r, "g"] & \mathsf{C} \\
\mathsf{B} \arrow[dashed, ur, bend right] & \quad
\end{tikzcd}
\end{equation*}
commute.
Recall the \textit{Yoneda embedding}, the functor $\yoneda \colon \mathsf{A} \to \Fun(\mathsf{A}^\op, \Set)$ induced by the assignment
\[
a \mapsto \big( a^\prime \mapsto \mathsf{A}(a^\prime, a) \big).
\]
A Kan extension along the Yoneda embedding is referred to as a \textit{Yoneda extension}.

\subsection{Simplicial chains}

We denote the \textit{simplex category} by $\simplex$, the category of \textit{simplicial sets} by $\sSet = \Fun(\stspx{n}\op, \Set)$ and the standard $n$-simplex by $\stspx{n}$.
As usual, we denote an element in $\stspx{n}_m$ by a non-decreasing tuples $[v_0, \dots, v_m]$ with $v_i \in \{0, \dots, n\}$.
The \textit{Cartesian product} of simplicial sets is defined ...


We denote the functor of \textit{(normalized) chains} by $\schains \colon \sSet \to \Ch$.
We omit the superscript $\simplex$ from either of these if no confusion may result from doing so.

The \textit{Alexander--Whitney coalgebra} functor is defined by a Kan extension argument based on the following natural maps.
For any $n \in \N$, define $\epsilon \colon \chains(\stspx{n}) \to \k$ by
\[
\epsilon \big( [v_0, \dots, v_q] \big) = \begin{cases} 1 & \text{ if } q = 0, \\ 0 & \text{ if } q>0, \end{cases}
\]
and $\Delta \colon \chains(\stspx{n}) \to \chains(\stspx{n})^{\otimes2}$ by
\[
\Delta \big( [v_0, \dots, v_q] \big) = \sum_{i=0}^q [v_0, \dots, v_i] \otimes [v_i, \dots, v_q].
\]

\subsection{Multisimplicial chains}

Let us consider an arbitrary non-negative integer~$k$.
The \textit{multisimplex category} $\msimplex{k}$ is the $k$-fold Cartesian product of the simplex category $\simplex$.
We denote the category of presheaves over $\msimplex{k}$ by $\sSet^{\times k}$ and refer to its objects as \textit{$k$-fold multisimplicial sets}.
This notation is justified by the isomorphism
\[
\Fun \big( (\msimplex{k})^\op,\, \Set \big) \cong
\Fun(\simplex^\op,\, \Set)^{\times k}.
\]
The functor of \textit{multisimplicial chains}
\[
\chains^{\msimplex{k}} \colon \sSet^k \to \Ch
\]
is defined as the $k^\th$ tensor product of the functor of simplicial chains.
Explicitly,
\[
\chains^{\msimplex{k}} \defeq \big( \schains \big)^{\otimes k}.
\]


