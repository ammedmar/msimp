% !TEX root = ../msimplicial.tex

\section{Preliminaries}\label{s:preliminaries}

\subsection{Chain complexes}

Throughout this article $\k$ denotes a commutative and unital ring and we work over its associated closed symmetric monoidal category of differential (homologically) graded $\k$-modules $(\Ch, \ot, \k)$.
We refer to the objects and morphisms of this category as \textit{chain complexes} and \textit{chain maps} respectively.
We denote by $\Hom(C, C^\prime)$ the chain complex of $\k$-linear maps between chain complexes $C$ and $C^\prime$, and refer to the functor $\Hom(-, \k)$ as \textit{linear duality}.

\subsection{Presheaves}

Recall that a category is said to be \textit{small} if its objects and morphisms form sets.
We denote the category of small categories by $\Cat$.
Given categories $\sB$ and $\sC$ with $\sB$ small we denote their associated \textit{functor category} by $\Fun(\sB, \sC)$.
A category is said to be \textit{cocomplete} if any functor to it from a small category has a colimit.
If $\sA$ is small and $\sC$ cocomplete, then the (\textit{left}) \textit{Kan extension of $g$ along $f$} exists for any pair of functors $f$ and $g$ in the diagram below, and it is the initial object in $\Fun(\sB, \sC)$ making
\begin{equation*}
\begin{tikzcd}[column sep=normal, row sep=normal]
\sA \arrow[d, "f"'] \arrow[r, "g"] & \sC \\
\sB \arrow[dashed, ur, bend right] & \quad
\end{tikzcd}
\end{equation*}
commute.
A Kan extension along the \textit{Yoneda embedding}, i.e., the functor
\[
\yoneda \colon \sA \to \Fun(\sA^\op, \Set)
\]
induced by the assignment
\[
a \mapsto \big( a^\prime \mapsto \sA(a^\prime, a) \big),
\]
is referred to as a \textit{Yoneda extension}.
Objects in the image of the Yoneda embedding as are said to be \textit{representable}.

%For any functor $F \colon \sA \to \sB$ and object $b \in \sB$ the objects and morphisms of the \textit{slice category} $(F \downarrow b)$ are morphisms $F(a) \to b$ and triangles
%\[
%\begin{tikzcd}[column sep=5, row sep=2]
%	F(a) \arrow[rr, "F(f)"] \arrow[dr, bend right] & & F(a^\prime) \arrow[dl, bend left] \\
%	& b &
%\end{tikzcd}
%\]
%respectively.
%We omit the forgetful functor $(\sA \downarrow b) \to \sA$ when it is clear from the context.
%For any object $X$ of $\Fun(\sA^\op, \Set)$ we have
%\[
%X \cong \colim_{\yoneda \downarrow X} \yoneda.
%\]
