% !TEX root = ../msimplicial.tex

\section{Conventions and preliminaries} \label{s:preliminaries}

\subsection{Chain complexes }

Throughout this article $\k$ denotes a commutative and unital ring and we work over its associated closed symmetric monoidal category of differential (homologically) graded $\k$-modules $(\Ch, \ot, \k)$.
We refer to the objects and morphisms of this category as \textit{chain complexes} and \textit{chain maps} respectively. We denote by $\Hom(C, C^\prime)$ the chain complex of $\k$-linear maps between chain complexes $C$ and $C^\prime$, and refer to the functor $\Hom(-, \k)$ as \textit{linear duality}.

\anibal{Be more explicit about tensor product and the closed monoidal property}

\subsection{Category theory}

%A category is said to be \textit{Cartesian} if is a monoidal category whose monoidal structure is given by the category-theoretic product and whose unit is a terminal object.
%A Cartesian category $(\mathsf{A}, \times, \mathbb{1})$ is equipped with natural diagonal and augmentation maps $\Delta_a \colon a \to a \times a$ and $\varepsilon_a \colon a \to \mathbb{1}$.
%
%\anibal{maybe this notion is not needed}

Recall that a category is said to be \textit{small} if its objects and morphisms form sets.
We denote the Cartesian category of small categories by $\Cat$.

Given categories $\sB$ and $\sC$ we denote their associated \textit{functor category} by $\Fun(\sB, \sC)$.

A category is said to be \textit{cocomplete} if any functor to it from a small category has a colimit.
If $\mathsf{A}$ is small and $\mathsf{C}$ cocomplete, then the (left) \textit{Kan extension of $g$ along $f$} exists for any pair of functors $f$ and $g$ in the diagram below, and it is the initial object in $\Fun(\sB, \mathsf{C})$ making
\begin{equation*}
\begin{tikzcd}[column sep=normal, row sep=normal]
\mathsf{A} \arrow[d, "f"'] \arrow[r, "g"] & \mathsf{C} \\
\sB \arrow[dashed, ur, bend right] & \quad
\end{tikzcd}
\end{equation*}
commute.
A Kan extension along the \textit{Yoneda embedding}, i.e., the functor
\[
\yoneda \colon \mathsf{A} \to \Fun(\mathsf{A}^\op, \Set)
\]
induced by the assignment
\[
a \mapsto \big( a^\prime \mapsto \mathsf{A}(a^\prime, a) \big),
\]
is referred to as a \textit{Yoneda extension}.
We refer to the objects in $\Fun(\mathsf{A}^\op, \Set)$ as \textit{presheaves on $\mathsf{A}$} and to those in the image of the Yoneda embedding as \textit{representable presheaves}.

\subsection{Simplicial sets}

We denote the \textit{simplex category} by $\simplex$, the category of \textit{simplicial sets} by $\sSet = \Fun(\simplex^\op, \Set)$ and the \textit{standard $n$-simplex} by
\[
\simplex^n = \yoneda \big( [n] \big)
\qquad \text{with} \qquad
\simplex^n_m = \simplex^n \big( [m] \big).
\]
As usual, we denote an element in $\stspx{n}_m$ by a non-decreasing tuple $[v_0, \dots, v_m]$ with $v_i \in \{0, \dots, n\}$ with \textit{face} and \textit{degenerate maps} denoted by
\begin{align*}
\face_i [v_0, \dots, v_m] & = [v_0, \dots, \widehat v_i, \dots, v_m], \\
\dege_i [v_0, \dots, v_m] & = [v_0, \dots, v_i, v_i, \dots, v_m].
\end{align*}

The \textit{Cartesian product} $X \times Y$ of simplicial sets is defined using the diagonal in small categories $\simplex \to \simplex \times \simplex$.
It makes $\sSet$ into a symmetric monoidal category.

\subsection{Simplicial chains}

The functor of (normalized) \textit{chains}
\[
\schains \colon \sSet \to \Ch,
\]
is defined as the Yoneda extension of the functor defined on representable simplicial sets by
\[
\schains(\simplex^n)_m =
\frac{\k\{\simplex^n_m\}}{\bigoplus \k\{\mathrm{img}(\dege_i)\}},
\qquad
\bd = \sum (-1)^i \face_i.
\]
We omit the superscript $\simplex$ from the notation $\schains$ if no confusion may result from doing so.

The functor of \textit{simplicial cochains} is defined through linear duality.

\subsection{EZ-AW contraction}

The functor of chains is not monoidal in general, but
it is bilax monoidal with natural transformations
\[
\begin{split}
\EZ & \colon \chains(X_1) \ot \chains(X_2) \to \chains(X_1 \times X_2), \\
\AW & \colon \chains(X_1 \times X_2) \to \chains(X_1) \ot \chains(X_2),
\end{split}
\]
known respectively as \textit{Eilenberg--Zilber} and \textit{Alexander--Whitney maps}.
These satisfy that $\EZ \circ \AW$ is the identity and that $\AW \circ \EZ$ is naturally chain homotopic to the identity.
Furthermore, they forms of associativity that make the extensions to maps
\[
\begin{split}
\EZ & \colon \chains(X_1) \ot \dotsb \ot \chains(X_k) \to \chains(X_1 \times \dotsb \times X_k), \\
\AW & \colon \chains(X_1 \times \dotsb \times X_k) \to \chains(X_1) \ot \dotsb \ot \chains(X_k).
\end{split}
\]
unambiguously defined.
%\begin{align*}
%\EZ \circ (\EZ \ot \, \id) &= \EZ \circ (\id \ot \EZ) \\
%(\AW \ot \, \id) \circ \AW &= (\id \ot \AW) \circ \AW
%\end{align*}
For future reference we give an explicit description of these maps.

\anibal{include the formulas}

\subsection{Alexander--Whitney coalgebra}

The \textit{Alexander--Whitney coalgebra} structure is defined by the following natural maps.
For any $n \in \N$, define $\epsilon \colon \chains(\triangle^n) \to \k$ by
\[
\epsilon \big( [v_0, \dots, v_q] \big) = \begin{cases} 1 & \text{ if } q = 0, \\ 0 & \text{ if } q>0, \end{cases}
\]
and $\Delta \colon \chains(\triangle^n) \to \chains(\triangle^n) \ot \chains(\triangle^n)$ by
\[
\chains(\simplex^n) \xra{\chains(\diag)} \chains(\simplex^n \times \simplex^n) \xra{\AW} \chains(\simplex^n) \ot \chains(\simplex^n)
\]
where $\diag$ is the diagonal in $\sSet$.
Explicitly,
\[
\Delta \big( [v_0, \dots, v_q] \big) =
\sum_{i=0}^q \, [v_0, \dots, v_i] \ot [v_i, \dots, v_q].
\]
The product induced on cochains is referred to as \textit{simplicial cup product}.
