
\section{Conventions and preliminaries} \label{s:preliminaries}

\subsection{Chain complexes }

Throughout this article $\k$ denotes a commutative and unital ring and we work over its associated closed symmetric monoidal category of differential (homological) graded $\k$-modules $(\Ch, \otimes, \k)$.
We refer to the objects and morphisms of this category as \textit{chain complexes} and \textit{chain maps} respectively. We denote by $\Hom(C, C^\prime)$ the chain complex of $\k$-linear maps between chain complexes $C$ and $C^\prime$, and refer to the functor $\Hom(-, \k)$ as \textit{linear duality}.

\subsection{Category theory}

A category is said to be \textit{Cartesian} if is a monoidal category whose monoidal structure is given by the category-theoretic product and whose unit is a terminal object.
A Cartesian category $(\mathsf{A}, \times, \mathbb{1})$ is equipped with natural diagonal and augmentation maps $\Delta_a \colon a \to a \times a$ and $\varepsilon_a \colon a \to \mathbb{1}$.

Recall that a category is said to be \textit{small} if its objects and morphisms form sets.
We denote the Cartesian category of small categories by $\Cat$.

Given categories $\sB$ and $\sC$ we denote their associated \textit{functor category} by $\Fun(\sB, \sC)$.

A category is said to be \textit{cocomplete} if any functor to it from a small category has a colimit.
If $\mathsf{A}$ is small and $\mathsf{C}$ cocomplete, then the (left) \textit{Kan extension of $g$ along $f$} exists for any pair of functors $f$ and $g$ in the diagram below, and it is the initial object in $\Fun(\sB, \mathsf{C})$ making
\begin{equation*}
\begin{tikzcd}[column sep=normal, row sep=normal]
\mathsf{A} \arrow[d, "f"'] \arrow[r, "g"] & \mathsf{C} \\
\sB \arrow[dashed, ur, bend right] & \quad
\end{tikzcd}
\end{equation*}
commute.
A Kan extension along the \textit{Yoneda embedding}, i.e., the functor
\[
\yoneda \colon \mathsf{A} \to \Fun(\mathsf{A}^\op, \Set)
\]
induced by the assignment
\[
a \mapsto \big( a^\prime \mapsto \mathsf{A}(a^\prime, a) \big),
\]
is referred to as a \textit{Yoneda extension}.
We refer to the objects in $\Fun(\mathsf{A}^\op, \Set)$ as \textit{presheaves on $\mathsf{A}$} and to those in the image of the Yoneda embedding as \textit{representable presheaves}.

\subsection{Simplicial chains}

We denote the \textit{simplex category} by $\simplex$, the category of \textit{simplicial sets} by $\sSet = \Fun(\simplex^\op, \Set)$ and the \textit{standard $n$-simplex} $\yoneda\big([n]\big)$ by $\stspx{n}$.
As usual, we denote an element in $\stspx{n}[m]$ by a non-decreasing tuples $[v_0, \dots, v_m]$ with $v_i \in \{0, \dots, n\}$.
The \textit{Cartesian product} of simplicial sets is defined ...

We denote the functor of (normalized) \textit{chains} by $\schains \colon \sSet \to \Ch$.
We omit the superscript $\simplex$ from the notation if no confusion may result from doing so.

The \textit{Alexander--Whitney coalgebra} functor is defined by a Yoneda extension based on the following natural maps.
For any $n \in \N$, define $\epsilon \colon \chains(\stspx{n}) \to \k$ by
\[
\epsilon \big( [v_0, \dots, v_q] \big) = \begin{cases} 1 & \text{ if } q = 0, \\ 0 & \text{ if } q>0, \end{cases}
\]
and $\Delta \colon \chains(\stspx{n}) \to \chains(\stspx{n})^{\otimes2}$ by
\[
\Delta \big( [v_0, \dots, v_q] \big) = \sum_{i=0}^q [v_0, \dots, v_i] \otimes [v_i, \dots, v_q].
\]

\subsection{Multisimplicial chains}

Let us consider an arbitrary non-negative integer~$k$.
The \textit{multisimplex category} $\msimplex{k}$ is the $k$-fold Cartesian product of the simplex category $\simplex$.
We denote the category of presheaves over $\msimplex{k}$ by $\sSet^{\times k}$ and refer to its objects as \textit{$k$-fold multisimplicial sets}.
This notation is justified by the isomorphism
\[
\Fun \big( (\msimplex{k})^\op,\, \Set \big) \cong
\Fun(\simplex^\op,\, \Set)^{\times k}.
\]
Furthermore, the representable multisimplicial sets are given by Cartesian products of representable simplicial sets.

The functor of \textit{multisimplicial chains}
\[
\mchainsk \colon \sSet^k \to \Ch
\]
is defined as the $k^\th$ tensor product of the functor of simplicial chains.
Explicitly,
\[
\mchainsk \defeq \big( \schains \big)^{\otimes k}.
\]
We denote the elements in the basis of $\mchainsk(\simplex^{n_1} \times \dots \times \simplex^{n_k})$ by $x_1 \otimes \dots \otimes x_k$ with $x_i \in \chains(\simplex^{n_i})$.

