% !TEX root = ../msimp.tex

\subsection{Table completion}\label{ss:table completion}

It is desirable to have a direct $\sym_r$-equivariant quasi-isomorphism between these algebraic models.
Two filtration preserving quasi-isomorphisms were constructed by Berger--Fresse
\[
\TR \colon \chains\BE(r) \to \chains\sur(r)
\quad \text{and} \quad
\TC \colon \chains\sur(r) \to \chains \BE(r).
\]
The first one, introduced in \cite[1$\cdot$3]{berger2004combinatorial}, is not a coalgebra map, as the reader familiar with its definition can easily verify.
We will focus on the second one which was introduced in \cite{berger2002prismatic} and termed \textit{table completion}.
We will construct a factorization up to signs
\[
\TC \colon \chains\sur(r) \xra{\EZ} \chains\sur(r)^\diag \xra{\chains(\tc)} \chains \BE(r),
\]
where the second map is induced from a simplicial map defined below.
This factorization proves that $\TC$ is a coalgebra map since both factors are.
We warn the reader that since $\EZ$ does not respect the $E_\infty$-coalgebra structure,
neither does $\TC$.
For example, we have
\[
\Delta_1 \circ \TC\big(12312) \neq \TC^{\otimes 2} \circ \, \Delta_1 \big(12312\big).
\]

Let us give a more explicit description of the simplicial set $\sur(r)^\diag$.
Each $m$-simplex corresponds to a map
\[
f \colon \set{1,\dots,rm+r} \to \set{1,\dots,r},
\]
satisfying that the cardinality of $f^{-1}(\ell)$ is $m+1$ for each $\ell \in \set{1,\dots,r}$.
We represent this simplex by the sequence $f(1) \dotsb f(rm+r)$.
The $i^\th$ face and degeneracy maps act on it respectively by removing or doubling the $i^\th$ occurrence of each $\ell \in \set{1,\dots,r}$ in $f(1) \dotsb f(rm+r)$.

The restriction to the diagonal defines a $\CG(r)$-filtration of $\sur(r)^\diag$ and a nested sequence
\[
\sur_1(r)^\diag \subset \sur_2(r)^\diag \subset \dotsb
\]
on $\sur(r)^\diag$ that is preserved by the action of $\sym_r$ on $\sur(r)^\diag$.
In terms of cellular $\CG(r)$-decompositions, given $(\mu,\sigma) \in \CG(r)$ then $f \in \sur(r)^\diag_{(\mu,\sigma)}$ if for each $i<j$, either $i$ and $j$ alternate strictly less than $\mu_{ij}$ times in the sequence $f_{ij}$, or they do so exactly $\mu_{ij}$ times and the ordering formed by the first occurrences of $i$ and $j$ in $f_{ij}$ agrees with $\sigma_{ij}$.

Since the complexity of an element is unchanged by degeneracy maps, it can easily be seen that $\EZ \colon \chains\sur(r) \to \chains\sur(r)^\diag$ preserves $\CG(r)$-filtrations.

Let us now define the simplicial map $\tc$.
For $f$ as above, let
\[
\tc(f) = (\sigma_0,\dots,\sigma_m)
\]
with $\sigma_j$ represented by the subsequence of $f$ containing the $(j+1)^{\mathrm{st}}$ occurrence of each $\ell \in \{1,\dots,r\}$.
For example, we have
\[
\tc(122333112) = (123,231,312).
\]

For every surjection $f\in \sur(r)_{m_1,\dots,m_r}$, as described in \cite[\S1.2.2]{berger1997confspacemodel}, there is a collection of elements $(k_0 ,\dots , k_{m-1})$, called \textit{caesuras}, where $m=m_1+\dots + m_r$ and $k_j \in \{1,\dots r\}$, $j\in\{0,\dots, m-1\}$. The caesuras are the elements of the sequence representing $f$ which are not the last occurrence of a value $k= 1,\dots, r$. The caesuras of
$f$ define a collection of morphisms $\pi_\ell \colon [m] \to [m_\ell]$, $\ell\in\{1,\dots,r\}$ such that for $j \in \set{0,\dots,m-1}$,  $\pi_{k_j}(j+1) = \pi_{k_j}(j)+1$ and $\pi_\ell(j+1) = \pi_\ell(j)$ for all $\ell \neq k_j$. We can interpret these morphisms geometrically, as in subsection \ref{ss:eilenber-zilber}. Then the collection $(\pi_1, \dots, \pi_r)$ defines a permutation $\vartheta_f\in \sym_m$ such that $\vartheta_f(j+1)=a_1 +\dots+ a_{k_j -1} + \pi_{k_j}(j+1)$, for $ j\in\{0,\dots, m-1\}$.

\begin{theorem}\label{thm:tc-decomposition}
	The simplicial map $\tc \colon \sur(r)^\diag \to \BE(r)$ satisfies
	\[
	\TC(f) = sign(\vartheta_f)(\chains(\tc) \circ \EZ )(f)
	\]
	and induces a weak equivalence
	\[
	\tc_d \colon \sur_d(r)^\diag \to \BE_d(r)
	\]
	for every $r,d \in \N$.
\end{theorem}

Before providing the proof of this theorem we briefly recall the definition of $TC$ from \cite[]{berger2002prismatic}. For each $f\in \sur(r)_{m_1,\dots,m_r}$ there is a certain map of simplicial sets $$\tau_f:\msimplex{m_1}{m_r}\to \BE(r)$$ Observe that a map into $\BE(r)$ is completely determined by its restriction to the $0$-simplexes. The $0$-simplexes of $\msimplex{m_1}{m_r}$ are $r$-tuples
$(n_1,\dots,n_r)$ such that $0 \leq n_i \leq m_i$,
for $i=1,\dots,r$. The map $\tau_f$ is determined by
the requirement that the permutation $\tau_f(n_1,\dots,n_r)$ is the subsequence of $f$ where one picks the $(n_i+1)^{st}$ entry of the value $i$, for each $i=1,\dots,r$.

As seen in subsection \ref{ss:eilenber-zilber}
each maximal non-degenerate simplex  $$\sigma\in (\msimplex{m_1 }{m_r})_m $$ is  determined by a
$(m_1,\dots,m_r)$-shuffle, and equivalently by a sequence $k_i\in \{1,\dots ,r \}$, $i=0,\dots , m-1$.
The \textit{fundamental simplex} of $f$ is the
maximal non-degenerate simplex corresponding to the  sequence $(k_0,\dots, k_{m-1})$ of the caesuras of $f$.
The homomorphism $TC(f)$ is defined as the sum, with signs, of the simplexes
$\tau_f(\sigma)$, where $\sigma$ runs
over all maximal non-degenerate simplices of the domain. The sign of each summand is positive if and only if the natural orientation of the corresponding simplex agrees with that of the fundamental simplex.

\begin{proof}[Proof of \Cref{thm:tc-decomposition}]
	It is clear that $tc$ is a simplicial map.
	Let
	\[
	\hat{f} \colon \simplex^{m_1,\dots,m_r} \to Sur
	\]
	be the map sending the only non-degenerate $(m_1,\dots,m_r)$-multisimplex $s$ to $f$. We claim that $$\tau_f = tc \circ \hat{f}^D$$ where we identify $$(\simplex^{m_1,\dots,m_r})^D \cong \msimplex{m_1}{m_r}$$  %reference  earlier
	Namely $\hat{f}^D$ coincides with $\tau_f$ on 0-simplexes and $tc_0$ is the identity.
	Observe that $EZ(s)$ is the signed sum of all maximal non-degenerate simplices $\sigma$ where the sign takes care of the orientation.
	By naturality $N(\hat{f}^D)$ sends $EZ(s)$ to $EZ(f)$ and so
	\[
	TC(f) = \pm N(\tau_f)(\EZ(s)) = \pm N(\tc)(N(\hat{f}^D)(\EZ(s)) = \pm N(\tc) (\EZ(f)).
	\]
	Comparing with the definition of $TC$ we see that the sign in the equation is exactly the sign of $\theta_f$.

	We now prove the filtration compatibility of $tc$. Let $f\in \sur(r)^{D}$ be an $m$-simplex and denote $$tc(f)=w=(w_{0},\dots,w_{m})\in \BE(r)_{m}$$ Suppose that in $f_{ij}$ $i$ and $j$ alternate $\mu_{ij}$ times, and in $w_{ij}$ $i$ and $j$ alternate $\mu_{ij}'$ times.
	We will prove that
	$\mu_{ij}' \leq \mu_{ij}$.

	Suppose without loss of generality that $i$ occurs before $j$ in $f_{ij}$, that starts with $i$ repeated $h$ times, followed by $j$ repeated $l$ times, and then by $i$ again (or terminating).
	If the sequence terminates then $h=l=k$ and $\mu_{ij}=\mu_{ij}'=0$.
	For $h <l$ let $\bar{f}_{ij}$ be the subsequence of $f_{ij}$ obtained taking out the first $h$ values of $i$ and of $j$.
	Let $\bar{\mu}_{ij}$ be the number of variations in $\bar{f}_{ij}$
	and $\bar{\mu}'_{ij}$ the number of variations in $tc(\bar{f}_{ij})$.

	Then $\bar{f}_{ij}$ starts with $j$, $\mu_{i,j}=1+\bar{\mu}_{ij}$, and $\mu'_{ij}=1+\bar{\mu}'_{ij}$.

	If $h \geq l$ and $l<r$ let $\bar{f}_{ij}$ be the subsequence of $f_{ij}$ obtained taking out the first $l$ values of $i$ and $j$.
	Then $\bar{f}_{ij}$	starts with $i$, $\mu_{i,j}=1+\bar{\mu}_{ij}$, and $\mu'_{ij}=\bar{\mu}'_{ij}$.
	By induction on the length of sequences, we obtain that $\mu_{ij}'\le \mu_{ij}$.
	Moreover, $tc$ is compatible with orderings: the first occurrences of $i$ and $j$ form the ordering $(w_1)_{ij}$ that is the first permutation of $tc(f_{ij})$.
	This concludes the proof of filtration compatibility.

	Regarding the maps $\tc_{d} \colon \sur_{d}(r)^{D}\rightarrow \BE(r)_{d}$ we can express source and target as (homotopy) colimits of contractible simplicial sets along the poset of complete graphs,
	so we have a commutative diagram
	\begin{equation*}
		\begin{tikzcd}
			hocolim_{\CG_d(r)}\sur_{(\mu,\sigma)}(r)^{D} \arrow[r] \arrow[d]&
			hocolim_{\CG_d(r)}\BE(r)_{(\mu,\sigma)} \arrow[d] \\
			\sur_{d}(r)^{D}\arrow[r] & \BE(r)_{d}&
		\end{tikzcd}
	\end{equation*}
	Where the vertical and top arrows
	are weak equivalences, and so the bottom map is a weak equivalence.
\end{proof}