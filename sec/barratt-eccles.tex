\subsection{Simplicial model}

We recall that the Barratt-Eccles simplicial sets $\mathcal{W}\Sigma_k$ is defined levelwise by $\mathcal({W}\Sigma_k)_i=(\Sigma_k)^{i+1}$.
The face $d_{j}$ removes the $(j+1)$-st permutation, and the degeneracy $s_j$ doubles the $(j+1)$-st permutation.
There is an operad structure on the collection $\mathcal{W} \Sigma_k$.

There is a filtration of the $W \Sigma_k$ by simplicial subsets
$W_d \Sigma_k$.
Let us write elements of $\Sigma_k$ as a string containing all numbers from $1$ to $k$.
Then $(w_0,\dots,w_i) \in W_d \Sigma_k$ if for any pair of indices their order of appearance in a string swaps at most $d-1$ times, as we proceed from $w_0$ to $w_i$.
for example $y=(123,213,312) \in \mathcal{W}_3 \Sigma_3$ but $y \notin \mathcal{W}_2 \Sigma_3$
because the position of $1$ and $2$ swaps two times.

The symmetric group $\Sigma_k$ acts on
$\mathcal{W} \Sigma_k$ by diagonal action and the action preserves the filtration.

The following is well known, see for example \cite{BF}
\begin{proposition} \ref{be-real}
	The geometric realization $\bars{\mathcal{W}_d \Sigma_k}$ is $\Sigma_k$-equivariantly homotopy equivalent to $F_d(\R^k)$.
\end{proposition}

The (normalized) chain complex of $\mathcal{W}_d\Sigma_k$ is called the Barratt-Eccles chain complex $$BE_d(k):=N(\mathcal{W}_d\Sigma_k)$$
the collection of these complexes forms the Barratt-Eccles operad $BE_d$.
\cite{BF}
There is a filtration
$$BE_0 \subset BE_1 \subset BE_2 \dots \subset BE$$

Similarly as for proposition \ref{sur-model} we have the following.

\begin{proposition}
	There is a  $\Sigma_k$-equivariant quasi-isomorphism
	of $E_\infty$-coalgebras between $BE_d(k)$ and
	$S^{(1)}(F_k(\R^d))$.
\end{proposition}

\begin{proof} \label{be-model}
	The required quasi-isomorphism follows from proposition \ref{be-real} and the equivariant quasi-isomorphism $BE_d(k) \simeq S^{(1)}(|W_d\Sigma_k|)$. %counit of quillen adjunction
\end{proof}

\subsection{A comparison morphism}

In this section we construct an explicit combinatorial map between the surjection complex and the Barratt-Eccles complex, that respects filtrations, inducing weak equivalences of subcomplexes.
Furthermore it upgrades a map previously defined on the chain level by Berger and Fresse: they constructed in \cite{BFsmall}
an operad map $TR:BE \to \chi$ preserving the filtrations and a collection of $\Sigma_k$-equivariant
chain maps $TC(k):\chi(k) \to BE(k)$ preserving filtrations
that are right inverses of $TR(k)$ but do not form an operad map.
We warn the reader that neither $TR$ nor $TC$ preserve the $E_\infty$-coalgebra structure, but
$TC$ preserves the $E_1$-coalgebra structure.

We claim that $TC$ is induced by a map of simplicial sets
$$tc: Sur(k)^D \to \mathcal{W}\Sigma_k$$ that we define.
\begin{definition}
	Let $s$ be an $i$-simplex  of the diagonal $Sur(k)^D$, i.e. a sequence containing any value in $\{1,\dots,k\}$ exactly $i+1$ times.
	Then $tc(s):=(\sigma_0,\dots,\sigma_i)$ is a sequence of permutations where $\sigma_j$ is the subsequence of $s$ containing the $(j+1)$-st occurrence of each value in $\{1,\dots,k\}$.
\end{definition}

For example
$$tc(122333112)=(123,231,312)$$

\begin{proposition}
	The homomorphism $TC:\chi(k) \to BE(k)$ by Berger-Fresse satisfies
	$$TC=N(tc) \circ \ez$$ where $\ez:N(Sur(k)) \to
	N(Sur(k)^D)$ is the Eilenberg-Zilber map.
\end{proposition}

\begin{proof}
	By inspection of the definition in \cite{BFsmall}.%more? signs? which convention?
\end{proof}

%Also $\mathcal{W}$ is filtered \cite{BFsmall}, i.e.
%there is a family of nested simplicial sets %$$\mathcal{W}_1\Sigma_k \subset \mathcal{W}_2 %\Sigma_k
% \subset \dots $$ such that
%$\mathcal{W}\Sigma_k={\rm colim}_d %\mathcal{W}_d\Sigma_k$, that form a filtration by %operads
%$$\mathcal{W}_1 \subset \mathcal{W}_2 \subset \dots %\mathcal{W}$$

%The normalized chain functor defines
%$$BE_d(k):=N(\mathcal{W}_d \Sigma_k)$$ so that the %operads $BE_d$
%form a filtration of $BE$.

\begin{definition}[Filtration of the Barratt-Eccles simplicial set $\mathcal{W}\Sigma_k$]
	Fix an element  $w\in (\mathcal{W}\Sigma_k)_{d}$, $w=(w_{1},\dots , w_{d})$ with $w_{h}\in \Sigma_k$.
	For any pair $(i,j)$ with $i< j$, denote $w_{ij}=(w_{1,ij},\dots , w_{d,ij})$ where $w_{h,ij}$ is the subsequence of $w_{h}$ obtained omitting all the occurrences of elements different from $i$ and $j$.
	%	For example, let $(123,231,312) \in (\mathcal{W}\Sigma_3)_{3}$, we have $(123,231,312)_{12}=(12,21,12)$, $(123,231,312)_{23}=(23,23,32)$ and $(123,231,312)_{13}=(13,31,31 )$.
	The element $w$ belongs to $\mathcal{W}_{(\mu,\sigma)}\Sigma_k\subseteq \mathcal{W}\Sigma_k$ if for all pairs $(i,j)$, $i< j$ either along the sequence $w_{ij}$ $i$ and $j$ swap less than $\mu_{ij}$ times,
	%	, that is, as before,  the amount of times that the order of $i$ and $j$ change ,
	or they swap exactly $\mu_{ij}$ times and the first permutation $w_{1,ij}$ is equal to $\sigma_{ij}$.
	Notice that
	\begin{equation*}
		\label{def}
		\mathcal{W}_{n}\Sigma_k=\bigcup_{\max_{i<j} (\mu_{ij})< n} \mathcal{W}_{(\mu,\sigma)}\Sigma_{k}
	\end{equation*}
	\\
	\\
	%	Returning to the example above we have $\mu_{12}(123,231,312)=2$, $\mu_{23}(123,231,312)=1$ and $\mu_{13}(123,231,312)=1$ so that $(123,231,312)\in (\mathcal{W}_{(\mu,\sigma)}\Sigma_3)_{3}$
	%	\begin{itemize}
		%		\item if $\mu_{12}>2$, $\mu_{23}>1$ and $\mu_{13}>1$ and any $\sigma\in \Sigma_{3}$;
		%		\item if some $\mu_{ij}=\mu_{ij}(w)$ and the first permutation of $w$ is equal to $\sigma_{ij}$.
		%	\end{itemize}
	%	In our example we obtain that, by equation \ref{def},
	%	$(123,231,312)\in (\mathcal{W}_{3}\Sigma_3)_{3}$
	%	\\
	%	\\
	%	We have obtained the filtration $$\mathcal{W}_1\Sigma_k \subset \mathcal{W}_2 \Sigma_k
	%	\subset \mathcal{W}_3 \Sigma_k
	%	\subset \dots $$
	%	that is the other one we need in $tc$ definition.
\end{definition}

% In both cases, to see even the role of permutations we'll use always pairs $(\mu,\sigma)$ with $\mu$ exactly the collection of variation numbers obtained from the initial element so that the permutation is uniquely determined, and $(\mu,\sigma)$ is minimal in the poset.
% \\
% In our examples
% \begin{itemize}
	% 	\item $31231\in Sur_{(\mu,\sigma)}(3)_{2,1,2} \subseteq Sur_{3}(3)_{2,1,2}$ with $(\mu_{12},\mu_{13},\mu_{23})=(1,2,1)$ and $\sigma=(312)$
	% 	\item $(123,231,312)\in (\mathcal{W}_{(\mu,\sigma)}\Sigma_3)_{3} \subset (\mathcal{W}_{3}\Sigma_3)_{3}$ with $(\mu_{12},\mu_{13},\mu_{23})=(2,1,1)$ and $\sigma=(123)$
	% \end{itemize}
%	We observe that the simplicial map $tc$ respects the filtration, sending
%	$Sur_d(k)^D$ to $\mathcal{W}_d\Sigma_k$
%	\\
%	\begin{example}
	%	We are givine some example of the compatibility of $tc$ to visualize why it works in general.
	%	Let $122333112 \in Sur(3)^{D} $
	%	\begin{itemize}
		%		\item $(122333112)_{12}=122112\rightarrow \mu_{12}=2$
		%		\item $(122333112)_{13}=133311\rightarrow \mu_{13}=1$
		%		\item $(122333112)_{23}=223332\rightarrow \mu_{23}=1$
		%	\end{itemize}
	%So $\mu=(\mu_{12},\mu_{13},\mu_{23})=(2,1,1)$
	%and the first occurrences of $1,2,3$ give us the permutation $\sigma=(123)$ $\Rightarrow$ $122333112 \in Sur_{((2,1,1),(123))}(3)^{D}$ $\Rightarrow$ $122333112 \in Sur_{3}(3)^{D}$.
	%Now applying $tc$ we obtain $tc(122333112)=(123,231,312)\in (\mathcal{W}\Sigma_3)_{3} $
	%\begin{itemize}
	%\item $(123,231,312)_{12}=(12,21,12)\rightarrow \mu_{12}=2$
	%\item $(123,231,312)_{13}=(13,31,31)\rightarrow \mu_{13}=1$
	%\item $(123,231,312)_{23}=(23,23,32)\rightarrow \mu_{23}=1$
	%\end{itemize}
	%So $\mu=(\mu_{12},\mu_{13},\mu_{23})=(2,1,1)$
	%and the first permutation in $(123,231,312)$ give us the permutation $\sigma=(123)$ $\Rightarrow$ $(123,231,312) \in (\mathcal{W}_{((2,1,1),(123))}\Sigma_{3})_{3}$ $\Rightarrow$ $(123,231,312) \in (\mathcal{W}_{3}\Sigma_{3})_{3}$.
	%
	%
	%
	%	\end{example}

\begin{lemma}[Compatibility of $tc$]
	We have that the map $tc:Sur(k)^{D}\rightarrow \mathcal{W}\Sigma_{k}$ is compatible with the filtrations, in the sense that
	$$tc(Sur_{(\mu,\sigma)}(k)^{D})\subseteq \mathcal{W}_{(\mu,\sigma)}\Sigma_{k}$$
	and in particular
	$$tc(Sur_{n}(k)^{D})\subseteq \mathcal{W}_{n}\Sigma_{k}$$
\end{lemma}

\begin{proof}
	Let  $f\in Sur(k)^{D}$ be an $m$-simplex and denote
	$$tc(f)=w=(w_{1},\dots,w_{m+1})\in (\mathcal{W}\Sigma_{k})_{m+1}$$
	Suppose that in $f_{ij}$ $i$ and $j$ alternate $\mu_{ij}$ times, and in $w_{ij}$ $i$ and $j$ alternate  $\mu_{ij}'$ times.
	We will prove that
	$\mu_{ij}' \leq \mu_{ij}$.

	Suppose without loss of generality that $i$ occurs before $j$ in $f_{ij}$, that starts with $i$ repeated $h$ times, followed by $j$ repeated $l$ times, and
	then by $i$ again (or terminating).
	If the sequence terminates then $h=l=k$ and $\mu_{ij}=\mu_{ij}'=0$.
	For $h <l$
	let $\bar{f}_{ij}$ be the subsequence of $f_{ij}$
	obtained taking out the first $h$ values of $i$ and of $j$.
	Let $\bar{\mu}_{ij}$ be the number of variations of $i$ and $j$ in $\bar{f}_{ij}$
	and $\bar{\mu}'_{ij}$ the number of variations of $i$ and $j$ in $tc(\bar{f}_{ij})$.

	Then $\bar{f}_{ij}$
	starts with $j$,
	$\mu_{i,j}=1+\bar{\mu}_{ij}$,
	and $\mu'_{ij}=1+\bar{\mu}'_{ij}$.

	If $h \geq l$ and $l<k$
	let $\bar{f}_{ij}$ be the subsequence of $f_{ij}$ obtained taking out the first $l$ values of $i$ and $j$.
	Then $\bar{f}_{ij}$
	starts with $i$,
	$\mu_{i,j}=1+\bar{\mu}_{ij}$,
	and $\mu'_{ij}=\bar{\mu}'_{ij}$.
	By induction on the length of sequences we obtain that
	$\mu_{ij}'\le \mu_{ij}$.\\
	Moreover $tc$ is compatible with orderings: the first occurrences of $i$ and $j$ form the ordering $w_{1,ij}$ that is the first permutation of $tc(f_{ij})$.
	This concludes the proof.
\end{proof}

\begin{corollary} \label{cor-comparing}
	The map $tc_{n}:Sur_{n}(k)^{D}\rightarrow \mathcal{W}_{n}\Sigma_{k}$ is a weak equivalence.
\end{corollary}

\begin{proof}
	We can express source and target as colimits of contractible %ok?
	simplicial sets along the poset of complete graphs,
	so we have a commutative diagram
	\begin{equation*}
		\begin{tikzcd}
			hocolim_{(\mu,\sigma)}Sur_{(\mu,\sigma)}(k)^{D} \arrow[r] \arrow[d]& hocolim_{(\mu,\sigma)}\mathcal{W}_{(\mu,\sigma)}\Sigma_{k} \arrow[d] \\
			Sur_{n}(k)^{D}\arrow[r] & \mathcal{W}_{n}\Sigma_{k}&
		\end{tikzcd}
	\end{equation*}
	Where the vertical and top arrows
	are weak equivalences, and so the bottom map is a weak equivalence.
\end{proof}
We remark that $tc$ is compatible with the equivalences in propositions \ref{sur-real} and \ref{be-real} since they are proved to be equivalences by a similar method.


%Let  $F_k(\R^{d})$ be the ordered configuration %space of $k$-tuples of points in  $\R^{d}$.

%\begin{proposition}
%The geometric realizations satisfy
%$$|Sur_d(k)| \simeq |W_d(k)|  \simeq F_k(\R^{d})$$
%\end{proposition}

%\begin{proof}
%The configuration space is homotopy equivalent to %the homotopy colimit of contractible spaces along %the same poset as above, as shown by McClure and %Smith \cite{MS}. %this is originally Berger's %criterion..
%\end{proof}