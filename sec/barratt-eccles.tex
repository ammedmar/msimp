% !TEX root = ../msimp.tex

\subsection{Simplicial model}\label{ss:simplicial model}

We recall the Barratt--Eccles simplicial set $\BE(r)$ defined for each $r\in\N$ that is equipped with a $\CG(r)$-filtration.
Applying the functor of chains to the nested sequence
\[
\BE_{1}(r) \subset \BE_{2}(r) \subset \dotsb.
\]
will provide the algebraic models
\[
\cE_{1}(r) \subset \cE_{2}(r) \subset \dotsb
\]
of configuration spaces studied by Berger and Fresse in \cite{berger2004combinatorial}.

The $n$-simplices of $\BE(r)$ are tuples of $n+1$ elements of the symmetric group $\sym_r$.
Its face and degeneracy maps are defined by removing and doubling elements respectively.
There is an operad structure on these simplicial sets, but we do not consider it here.

Next we recall a $\CG(r)$-filtration on $\BE(r)$.
For $i<j$ and $\sigma$ in $\sym_r$ let $\sigma_{ij}$ be the associated permutation in $\sym_2$.
Given $(\mu,\sigma) \in \CG(r)$ then an
element $w = (w_0,\dots w_n) \in \BE(r)_n$
$w \in \sur(r)_{(\mu,\sigma)}$ if for each $i<j$, the cardinality of $\set{\ell \mid (w_{\ell})_{ij} \neq (w_{\ell+1})_{ij}}$ is either less than $\mu_{ij}$ or equal to it and $(w_0)_{ij} = \sigma_{ij}$.

In particular $w$ has complexity $d$ or less if for each $i<j$
%the cardinality of $\set{\ell=1,\dots,n-1 \mid w_{\ell-1,ij} \neq w_{\ell,ij}}$ is less than $d$, i.e., 
the non-degenerate dimension of $w_{ij}=((w_0)_{ij},\dots,(w_n)_{ij})$ in $\BE(2)$ is $d$ or less for all $i<j$.
We notice that the action of $\sym_r$ on $\BE(r)$ preserves the nested sequence
$$\BE_1(r) \subset \BE_2(r) \subset \dots$$
For a proof that this is a $\CG(r)$-filtration we refer to Example 2.8 in \cite{berger1997confspacemodel}.
Please consult \cite{smith1989filtration,kashiwabara1993confcomplex,berger1997confspacemodel} for more details.
 
Applying the functor of singular chains to the zig-zag of \cref{p:zig-zag conf} produces a zig-zag of equivariant quasi-isomorphisms of $\UM$-coalgebras connecting $\schains\bars{\BE_d(r)}$ and $\schains\conf{r}{d}$.
Using the unit of the Quillen equivalence extends this zig-zag to one relating $\chains\BE_d(r)$ and $\schains\conf{r}{d}$, which can be combined with the zig-zag constructed in the previous subsection.
As announced in the introduction, this construction relates the chains on the multisimplicial model of configuration space and those the simplicial model via an explicit zig-zag of equivariant quasi-isomorphisms of $E_\infty$-coalgebras.

