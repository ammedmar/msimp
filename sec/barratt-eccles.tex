% !TEX root = ../msimp.tex

\subsection{Simplicial model}\label{ss:simplicial model}

We recall the Barratt--Eccles simplicial set $\BE(r)$ defined for each $r\in\N$ and equipped with a $\CG(r)$-filtration
\[
\BE_{1}(r) \subset \BE_{2}(r) \subset \dotsb.
\]
which, by applying the functor of chains to it, recovers an algebraic models
\[
\cE_{1}(r) \subset \cE_{2}(r) \subset \dotsb
\]
of configuration spaces.

The $n$-simplices of $\BE(r)$ are tuples of $n+1$ elements of the symmetric group $\sym_r$.
Its face and degeneracy maps are defined by removing and doubling elements respectively.
There is an operad structure on these simplicial sets, but we do not consider it here.

For $i<j$ and $\sigma$ in $\sym_r$ let $\sigma_{ij}$ be the associated permutation in $\sym_2$.
%An element $(w_1,\dots w_n) \in \BE(r)$ and $i<j$, let $w_{ij} = (w_{1,ij}, \dots, w_{d,ij})$ with $w_{\ell,ij}$ the subsequence of $w_\ell$ obtained by omitting all the occurrences of elements different from $i$ and $j$.
%	For example, let $(123,231,312) \in (\BE_3)_{3}$, we have $(123,231,312)_{12}=(12,21,12)$, $(123,231,312)_{23}=(23,23,32)$ and $(123,231,312)_{13}=(13,31,31 )$.
The element $(w_1,\dots w_n)$ in $\BE(r)_{n}$ belongs to $\BE_{(\mu,\sigma)}(r)$ if for all $i<j$ the cardinality of $\set{w_{\ell,ij} \neq w_{\ell+1,ij}}$ is either less than $\mu_{ij}$ or equal to it and $(w_1)_{ij} = \sigma_{ij}$.
Therefore, $(w_1,\dots w_n)$ is in $\BE_{d}(r)$ iff the cardinality associated to each $i<j$ is less than $d$.

We notice that the action of $\sym_r$ on $\BE(r)$ preserves this $\CG(r)$-filtration.
Please consult \cite{smith1989filtration,kashiwabara1993confcomplex,berger1997confspacemodel} for more details. In terms of cellular $\CG(r)$-decompositions, applying realization functor to the above $\CG(r)$-filtration results in a cellular $\CG_d(r)$-decomposition of each $\bars{\BE_{d}(r)}$, then we obtain by proposition \ref{p:berger} the homotopy equivalence
$$\conf{r}{d}\cong \bars{\BE_{d}(r)}.$$

Applying the functor of singular chains gives
$\schains\conf{r}{d} \to \schains\bars{\BE_{d}(r)}$
a quasi-isomorphisms of $\UM$-coalgebras and using the unit we get a zig-zag of quasi-isomorphisms of $\UM$-coalgebras
\begin{equation}\label{eq:simplicial zig-zag}
	\schains\conf{r}{d} \cong
	\schains\bars{\BE_d(r)} \leftarrow
	\chains \BE_d(r)= \cE_d(r).
\end{equation}
Combining this with the zig-zag constructed in the previous subsection yields the following.

\begin{theorem}
	The chains on the simplicial and multisimplicial models of a configuration space are related by an explicit zig-zag of quasi-isomorphism of $E_\infty$-coalgebras.
\end{theorem}