% !TEX root = ../msimp.tex

\subsection{Multisimplicial model}\label{ss:surjection model}

We define for each positive integer $r$ a multisimplicial sets $\sur(r)$ equipped with a $\CG(r)$-filtration
\[
\sur_{1}(r) \subset \sur_{2}(r) \subset \dotsb
\]
which, by applying the functor of chains to it, recovers the algebraic models
\[
\chi_1(r) \subset \chi_2(r) \subset \dotsb
\]
of configuration spaces developed by McClure--Smith \cite{mcluresmith2004geomodel}.

Spaces $Y_0^r$ homeomorphic to $\bars{\sur(r)}$ were studied as topological spaces in the work of McClure--Smith \cite{mcclure2003multivariable}.
The homeomorphism between $Y_0^r$ and $\bars{\sur(r)}$ is described explicitly in the appendix of \cite{salvatore2009deligne}.

Let $\sur(r)$ be the $k$-fold multisimplicial set that has as $(m_1,\dots,m_r)$-multisimplices the surjective maps
\[
f \colon \set{1,\dots,m+r} \to \set{1,\dots,r},
\]
where $m = m_1+\dots+m_r$, satisfying that the cardinality of $f^{-1}(\ell)$ is $m_\ell$ for each $\ell \in \set{1,\dots,r}$.
We represent this multisimplex by the sequence $f(1) \dotsm f(m+r)$.
The face and degeneracy maps $\face^\ell_j$ and $\dege^\ell_j$ act on it by respectively removing and doubling the $(j+1)^\th$ occurrence of $\ell$ in the sequence.

For $i<j$, let $f_{ij}$ be the subsequence of $f(1) \dotsm f(m+r)$ obtained by omitting all occurrences of elements different from $i$ and $j$.
The surjection $f$ belongs to $\sur_{(\mu,\sigma)}(r) \subset \sur(r)$ if for each $i<j$, either $i$ and $j$ alternate strictly less than $\mu_{ij}$ times in the sequence $f_{ij}$, or they do so exactly $\mu_{ij}$ times and the ordering formed by the first occurrences of $i$ and $j$ in $f_{ij}$ agrees with $\sigma_{ij}$.
Therefore, $f \in \sur_{d}(r)$ if the alternation number of each $f_{ij}$ is less than $d+1$.

We notice that the action of $\sym_r$ on $\sur(r)$ preserves this $\CG(r)$-filtration.
In terms of cellular $\CG(r)$-decompositions, applying realization functor to the above $\CG(r)$-filtration results in a cellular $\CG_d(r)$-decomposition of each $\bars{\sur_{d}(r)}$, then we obtain by proposition \ref{p:berger} the homotopy equivalence
$$\conf{r}{d}\cong \bars{\sur_{d}(r)}.$$

Applying the functor of singular chains gives
$\schains\conf{r}{d} \to \schains\bars{\sur_{d}(r)}$
a quasi-isomorphisms of $\UM$-coalgebras, which extends to the right with maps of the same kind
\[
\schains\bars{\sur_{d}(r)} \cong
\schains \bars{\sur_{d}(r)^{D}} \to
\chains(\sur_{d}(r)^D) \to
\chains(\cN^{(r)}(\sur_{d}(r)^D)) \leftarrow
\chains\sur_{d}(r)
\]
given respectively from the left by the homeomorphism between multisimplicial set and diagonal simplicial set realizations, the unit of the Quillen equivalence between simplicial sets and topological spaces, the comparison map of \cref{ss:inclusion}, and the unit of the Quillen equivalence between multisimplicial sets and simplicial sets.

As announced in the introduction, this construction relates the singular chains of a configuration space and the chains of its multisimplicial model as expressed in the following.

\begin{theorem}
	The singular chains of a configuration space and the chains of its multisimplicial model are related by an explicit zig-zag of quasi-isomorphism of $E_\infty$-coalgebras.
\end{theorem}