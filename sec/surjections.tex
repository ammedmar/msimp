\subsection{Multisimplicial model}\label{ss:surjection model}

We define for each positive integer $r$ a family of multisimplicial sets $\sur(r)$ equipped with a complete graph filtration
\[
 \sur_{1}(r) \subset \sur_{2}(r) \subset \dotsb
\]
which, by applying to it the functor of chains, recovers the algebraic models of configuration spaces developed by McClure--Smith \cite{mcluresmith2004geomodel}.

The geometric realizations $\bars{\sur(r)}$ appear in the work of McClure--Smith \cite{mcclure2003multivariable}, as explained in the appendix of \cite{salvatore2009deligne}.
\todo{@paolo: What do you mean?}

Let $\sur(r)$ be the $k$-fold multisimplicial set that has as $(m_1,\dots,m_r)$-multisimplices
the surjective maps
\[
f \colon \set{1,\dots,m+r} \to \set{1,\dots,r},
\]
where $m = m_1+\dots+m_r$, satisfying that the cardinality of $f^{-1}(\ell)$ is $m_\ell$ for each $\ell \in \set{1,\dots,r}$.
We represent this multisimplex by the sequence $f(1) \dotsm f(m+r)$.
The face and degeneracy maps $\face^\ell_j$ and $\dege^\ell_j$ act on it by respectively removing and doubling the $(j+1)^\th$ occurrence of $\ell$ in the sequence.
%For example,
%\begin{align*}
%	d^2_0(12321)&=1321, & s^1_0(121)&=1121, \\
%	d^2_1(12321)&=1231, & s^1_1(121)&=1211.
%\end{align*}
%Degenerate multisimplices are exactly the sequences containing two equal adjacent terms.
Next we define a cellular $\CG(r)$-decomposition on the geometric realization of $\sur(r)$, which induces a filtration
\[
\bars{\sur_1(r)} \subset \bars{\sur_2(r)} \subset \dotsb
\]
with each
\[
\bars{\sur_d(r)} = \, \bigcup_{\mathclap{\CG_d(r)}} \, \bars{\sur_{(\mu,\sigma)}(r)}
\]
equipped with a cellular $\CG_d(r)$-decomposition.
For $i<j$, let $f_{ij}$ be the subsequence of $f(1) \dotsm f(m+r)$ obtained by omitting all occurrences of elements different from $i$ and $j$.
%For example, let $31231 \in \sur(3)_{2,1,2}$, we have $(31231)_{12}=121$, $(31231)_{23}=323$ and $(31231)_{13}=3131$.
The surjection $f$ belongs to $\sur_{(\mu,\sigma)}(r) \subset \sur(r)$ if for each $i<j$, either $i$ and $j$ alternate strictly less than $\mu_{ij}$ times in the sequence $f_{ij}$, or they do so exactly $\mu_{ij}$ times and the ordering formed by the first occurrences of $i$ and $j$ in $f_{ij}$ agrees with $\sigma_{ij}$.
Therefore, $f \in \sur_{d}(r)$ if the alternation number of each $f_{ij}$ is at most $d+1$.
We notice that
%For example, $x = 12132 \in \sur_3(3)$ but $x \notin \sur_2(3)$ because the subsequence $1212$ has length 4.
the action of $\sym_r$ on $\sur(r)$ preserves the filtration.

Applying the functor of singular chains to the induced zig-zag of equivariant homotopy equivalences coming from \cref{p:berger} gives
\[
\schains\conf{r}{d} \leftarrow
\schains\bars{\CG_{d}(r)} \to
\schains\bars{\sur_{d}(r)},
\]
a zig-zag of equivariant quasi-isomorphisms of $\UM$-coalgebras, which extends to the right with maps of the same kind
\[
\schains\bars{\sur_{d}(r)} \cong
\schains \bars{\sur_{d}(r)^{D}} \to
\chains(\sur_{d}(r)^D) \to
\chains(\cN^{(r)}(\sur_{d}(r)^D)) \leftarrow
\chains\sur_{d}(r)
\]
given respectively from the left by the homeomorphism between multisimplicial set and diagonal simplicial set realizations, the unit of the Quillen equivalence between simplicial sets and topological spaces, the comparison map of \cref{ss:inclusion}, and the unit of the Quillen equivalence between multisimplicial sets and simplicial sets.

As announced in the introduction, this construction relates the singular chains of a configuration space and the chains of its multisimplicial model through an explicit zig-zag of equivariant quasi-isomorphism of $E_\infty$-coalgebras.