% !TEX root = ../msimp.tex

\subsection{Multisimplicial model}\label{ss:surjection model}

We define for each positive integer $r$ a multisimplicial set $\sur(r)$ equipped with a $\CG(r)$-filtration.
The functor of chains applied to the nested sequence
\[
\sur_{1}(r) \subset \sur_{2}(r) \subset \dotsb
\]
 will recover the algebraic models
\[
\chi_1(r) \subset \chi_2(r) \subset \dotsb
\]
of configuration spaces developed by McClure--Smith \cite{mcluresmith2004geomodel}.

Spaces $Y_0^r$ homeomorphic to $\bars{\sur(r)}$ were studied in the work of McClure--Smith \cite{mcclure2003multivariable}.
The homeomorphism between $Y_0^r$ and $\bars{\sur(r)}$ is described explicitly in the appendix of \cite{salvatore2009deligne}.

Let $\sur(r)$ be the $k$-fold multisimplicial set that has as $(m_1,\dots,m_r)$-multisimplices the surjective maps
\[
f \colon \set{1,\dots,m+r} \to \set{1,\dots,r},
\]
where $m = m_1+\dots+m_r$, satisfying that the cardinality of $f^{-1}(\ell)$ is $m_\ell$ for each $\ell \in \set{1,\dots,r}$.
We represent this multisimplex by the sequence $f(1) \dotsm f(m+r)$.
The face and degeneracy maps $\face^\ell_j$ and $\dege^\ell_j$ act on it by respectively removing and doubling the $(j+1)^\th$ occurrence of $\ell$ in the sequence.

Next we define a $\CG(r)$-filtration on $\sur(r)$.
For $i<j$, let $f_{ij}$ be the subsequence of $f(1) \dotsm f(m+r)$ obtained by omitting all occurrences of elements different from $i$ and $j$.
For $(\mu,\sigma) \in \CG(r)$
we say that 
$f \in \sur(r)_{(\mu,\sigma)}$
 if for each $i<j$, either $i$ and $j$ alternate strictly less than $\mu_{ij}$ times in the sequence $f_{ij}$, or they do so exactly $\mu_{ij}$ times and the ordering formed by the first occurrences of $i$ and $j$ in $f_{ij}$ agrees with $\sigma_{ij}$.
%reference of proof

The surjection $f$ has complexity $d$ or less if the alternation number of each $f_{ij}$ is less than $d+1$, i.e., if the non-degenerate dimension of $f_{ij}$ in $\sur(2)$ is $d$ or less for each $i<j$.
We notice that the action of $\sym_r$ on $\sur(r)$ preserves the nested sequence
$$\sur_1(r) \subset \sur_2(r) \subset \dots$$
For the proof that $|\sur(r)|$
has indeed an induced cellular $\CG(r)$-decomposition we refer to 
Lemma 14.8 in \cite{mcluresmith2004geomodel}.
Applying the functor of singular chains to the zig-zag of \cref{p:zig-zag conf} produces a zig-zag of equivariant quasi-isomorphisms of $\UM$-coalgebras connecting $\schains\bars{\sur_d(r)}$ and $\schains\conf{r}{d}$.
We can extend it using the following zig-zag of maps of the same kind
\[
\schains\bars{\sur_{d}(r)} \cong
\schains\bars{\sur_{d}(r)^{D}} \to
\chains(\sur_{d}(r)^D) \to
\chains(\cN^{(r)}(\sur_{d}(r)^D)) \leftarrow
\chains\sur_{d}(r).
\]
The first map is induced by the homeomorphism $\bars{\sur_{d}(r)} \cong \bars{\sur_{d}(r)^{D}}$, the second by the unit of the Quillen equivalence between simplicial sets and topological spaces, the third is the comparison map of \cref{ss:inclusion}, and last one is induced by the unit of the Quillen equivalence between multisimplicial sets and simplicial sets.

As announced in the introduction, this construction relates the chains on the multisimplicial model of configuration space and its singular chains via an explicit zig-zag of equivariant quasi-isomorphisms of $E_\infty$-coalgebras.