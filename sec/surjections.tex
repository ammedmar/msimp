\subsection{Multisimplicial model} \label{ss:multisimplicial model}

We define for each positive integer $k$ a family of multisimplicial sets $\sur(k)$ equipped with a complete graph filtration
\[
\sur(k,0) \subset \sur(k,1) \subset \sur(k,2) \subset \dotsb
\]
which, by applying to it the functor of chains, recovers the algebraic models of configuration spaces developed by McClure--Smith \cite{mcclure2003multivariable}.

Also the geometric realizations $\mathcal{F}(k) = \bars{\sur(k)}$ appear in the work by McClure-Smith \cite{MS}, as explained in the appendix of \cite{Deligne}.
\todo{@paolo: this sentence appears cryptic to me. Can you say what you mean? Maybe in the previous sentence we can say not algebraic models but ``space'' level.}

Let $\sur(k)$ be the $k$-fold multisimplicial set that has as $(m_1,\dots,m_k)$-multisimplices
the surjective maps
\[
f \colon \set{1,\dots,m+k} \to \set{1,\dots,k}
\]
where $m = m_1+\dots+m_k$ satisfying that the cardinality of $f^{-1}(\ell)$ is $m_\ell$ for each $\ell \in \set{1,\dots,k}$.
We represent this multisimplex by the sequence $f(1) \dotsm f(m+k)$.
The face and degeneracy maps $d^\ell_j$ and $s^\ell_j$ act on it by respectively removing and doubling the $(j+1)^\th$ occurrence of $\ell$ in the sequence.
For example,
\begin{align*}
	d^2_0(12321)&=1321, & s^1_0(121)&=1121, \\
	d^2_1(12321)&=1231, & s^1_1(121)&=1211.
\end{align*}
Degenerate multisimplices are exactly the sequences containing two equal adjacent terms.
A surjection belongs to $\sur(k,d)$ if any ordered subsequence with two distinct alternating values has length at most $d+1$.
For example, $x = 12132 \in \sur_3(3)$ but $x \notin \sur_2(3)$ because the subsequence $1212$ has length 4.
This is a complete graph filtration since...
\todo{@paolo: verify the conditions of complete graph filtration to get the (now commented out) proposition below}
The symmetric group $\Sigma_k$ acts on $\sur(k)$ by post-composition and the action preserves the filtration.

%\begin{proposition} \label{sur-real}
%	There is a $\Sigma_k$-equivariant homotopy equivalence
%	$\bars{\sur_d(k)} \simeq F_k(\R^d)$
%\end{proposition}
%
%\begin{proof}
%	The geometric realization $\bars{\sur_d(k)}$ appears in section 13 of \cite{MS} where it is denoted $Z_0^k$.
%	McClure and Smith define for each $d$ a topological operads $\mathcal{D}_d$ such that $\mathcal{D}_d(k)$ is $\Sigma_k$-homeomorphic to $\bars{\sur_d(k)} \times Tot(\Delta^*)$, where $Tot(\Delta^*)$ is contractible and equipped with trivial $\Sigma_k$-action.
%	This is explained in the appendix of \cite{cyclic} where the realization is constructed as a CW complex and denoted $\mathcal{F}_d(k)$.
%	McClure and Smith proceed to constructing a zig-zag of weak equivalences of operads between $\mathcal{D}_d$ and the little $d$-cubes operad
%	$\mathcal{C}_d$, that in particular gives a levelwise $\Sigma_k$-equivariant equivalence $\mathcal{D}_d(k) \simeq \mathcal{C}_d(k)$ for each $k$. It is well known that $\mathcal{C}_d(k)$ is $\Sigma_k$-equivariantly homotopy equivalent to $F_k(\R^d)$ and this concludes the proof.
%\end{proof}


%The functor of chains applied to these multisimplicial sets induces chain complex $\chi(k,d) \defeq \chains(\sur(k,d))$ with a filtration
%\[
%\chi(k,0) \subset \chi(k,0) \subset \chi(k,2) \subset \dotsb,
%\]
%which were considered by McClure and Smith \cite{MS}, %actually other paper
%who constructed an operad structure on the collections of these complexes, the {\it surjection operad} $\chi$.
%gives a filtration of the surjection operad by suboperads
%$$\chi_0 \subset \chi_1 \subset \chi_2 \subset \dots \subset \chi$$

\begin{theorem*}
	For any configuration space $\con(r,d)$ the maps in the following zig-zag are $\sym_r$-equivariant quasi-isomorphisms of $E_\infty$-coalgebras, or more precisely of $\UM$-coalgebras:
	\[
	\schains\con(r,d) \leftarrow \schains \cgr(r,d) \to \schains\bars{\sur(r,d)} \to \schains^{(r)}\bars{\sur(r,d)} \leftarrow \chains(\sur(r,d))
	\]

\end{theorem*}

\begin{proposition} \label{sur-model}
	$\chi_d(k)$ and $\rS{F_k(\R^d)}$ are $\Sigma_k$-equivariantly quasi-isomorphic $E_\infty$-coalgebras
\end{proposition}

\begin{proof}
	The $\Sigma_k$-equivariant homotopy equivalence of proposition \ref{sur-real} induces a $\Sigma_k$-equivariant quasi-isomorphism of $E_{\infty}$-coalgebras
	$$S^{(k)}(\bars{\sur_d(k)}) \simeq S^{(k)}(F_k(\R^d))$$.
	The weak equivalence %say that it is Quillen adjunction
	$\sur_d(k) \to Sing^{(k)}(\bars{\sur_d(k)})$ induces on the chain level a $\Sigma_k$-equivariant quasi-isomorphism of
	$E_\infty$-coalgebras
	between $\chi_d(k)$
	and $S^{(k)}(\bars{\sur_d(k)})$.
	Finally the map of lemma \ref{} gives a $\Sigma_k$-equivariant quasi-isomorphism $S^{(1)}(F_k(\R^d)) \to S^{(k)}(F_k(\R^d))$ of $E_\infty$-coalgebras.
\end{proof}
