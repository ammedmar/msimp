% !TEX root = ../msimplicial.tex

\section{Multisimplicial topology}

\anibal{say something here}

\subsection{Multisimplicial sets}

Let us consider an arbitrary non-negative integer~$k$.
The \textit{$k$-fold multisimplex category} $\msimplex{k}$ is the $k$-fold Cartesian product of the simplex category $\simplex$, with the convention that for $k = 0$ this is the trivial category.
The category of \textit{multisimplicial sets} is the coproduct
\[
\mSet = \coprod_{k \in \N} \mSet^{(k)}
\]
where $\mSet^{(k)}$ denotes the category of presheaves $\Fun \big( (\msimplex{k})^\op, \Set \big)$, whose objects are referred to as \textit{$k$-fold multisimplicial sets}.
The categories $\mSet^{(0)}$ and $\mSet^{(1)}$ are naturally equivalent to $\Set$ and $\sSet$ respectively.
The representable $k$-fold multisimplicial sets are denoted by
\[
\simplex^{n_1, \dots, n_k} =
\yoneda \big( [n_1] \times \cdots \times [n_k] \big)
\]
and we have
\[
\simplex^{n_1, \dots, n_k} \big( [m_1] \times \cdots \times [m_k] \big) \cong
\simplex^{n_1}_{m_1} \times \dots \times \simplex^{n_k}_{m_k}.
\]
Explicitly, a $k$-fold multisimplicial set $X$ consists of a collection of sets
\[
X_{m_1, \dots, m_k} =
X \big( [m_1] \times \cdots \times [m_k] \big)
\]
indexed by tuples of non-negative integers $(m_1, \dots, m_k)$ together with \textit{face maps}
\[
\face_i^j \colon
X_{m_1, \dots, m_j, \dots, m_k} \to
X_{m_1, \dots, m_j-1, \dots, m_k}
\]
and \textit{degeneracy map}
\[
\dege^j_i \colon X_{m_1, \dots, m_j, \dots, m_k} \to X_{m_1, \dots, m_j+1, \dots, m_k}
\]
for $1 \leq j \leq k$ and $0 \leq i \leq m_j$ such that, referring to $j$ as the \textit{direction} of these maps, two of them satisfy the simplicial identities when they have the same direction and commute when they do not.
An element of $X_{m_1, \dots, m_k}$ is called
$(m_1, \dots, m_k)$-\textit{multisimplex}.
A multisimplex is \textit{degenerate} if it is in the image of a degeneracy map.

\subsection{The pair $(-)^\diag \dashv \, \fM$} \label{ss:diagonal simplicial set}

The composition with the diagonal functor
\[
\simplex^\op \xra{\diag}
(\simplex^\op)^{\times k} \xra{\cong}
(\msimplex{k})^{\op}
\]
defines a functor
\[
(-)^{\diag} \colon \mSet \to \sSet
\]
explicitly defined on a $k$-fold multisimplicial set $X$ by
\begin{gather*}
X^{\diag}_m = X_{m, \dots, m},
\qquad
\face_i = \face_i^1 \circ \dots \circ \face_i^k,
\qquad
\dege_i = \dege_i^1 \circ \dots \circ \dege_i^k.
\end{gather*}
It is straightforward to verify that
\[
\big( \simplex^{n_1, \dots, n_k} \big)^{\diag} \cong
\simplex^{n_1} \times \dots \times \simplex^{n_k}
\]
as simplicial sets.

The functor $(-)^\diag \colon \mSet \to \sSet$ admits a right adjoint $\fM \colon \sSet \to \mSet$.
It is defined on a simplicial set $Y$ by
\[
\fM(Y)_{m_1, \dots, m_k} =
\sSet\big( \simplex^{m_1} \times \dots \times \simplex^{m_k}, \, Y \big)
\]
and on simplicial maps by composition.

\subsection{Shuffles and the fundamental simplex} \label{ss:shuffles and fundamental simplex}

Let $n = n_1 + \dots + n_k$.
An \textit{$(n_1, \dots, n_k)$-shuffle} $\sigma$ is an automorphism of $\set{0 + \dots + n-1}$ satisfying
\begin{gather*}
	\sigma(0) < \dots < \sigma(n_1-1), \\
	\sigma(n_1) < \dots < \sigma(n_1+n_2-1), \\
	\vdots \\
	\sigma(n_1+\dots+n_{k-1}) < \dots < \sigma(n_1+\dots+n_k-1).
\end{gather*}
The set of all of these -- which we denote $\sh(n_1, \dots, n_k)$ -- serves to parameterize the set of non-degenerate $n$-simplices of $\simplex^{n_1} \times \dots \times \simplex^{n_k}$ via the inclusion
\[
\begin{tikzcd}[column sep=tiny, row sep=0]
	\cI \colon &[-10pt]
	\sh(n_1, \dots, n_k) \arrow[r] &
	\big(\simplex^{n_1} \times \dots \times \simplex^{n_k}\big)_n \\ &
	\sigma \arrow[r, mapsto] &
	s_{V_1} [n_1] \times \dots \times s_{V_k} [n_k]
\end{tikzcd}
\]
where $s_{V_j} = s_{v_{n-n_j}^j} \dotsb \ s_{v_1^j}$ for $V_j = \set{v_1^j < \dots < v_{n-n_j}^j}$ and
\begin{gather*}
	V_1 = \set{0,\dots,n-1} \setminus \sigma^{-1} \set{0,\dots,n_1-1} \\
	V_2 = \set{0,\dots,n-1} \setminus \sigma^{-1} \set{n_1, \dots, n_1+n_2-1} \\
	\vdots \\
	V_k = \set{0,\dots,n-1} \setminus \sigma^{-1} \set{n_1+\dots+n_{k-1}, \dots, n_1+\dots+n_k-1}.
\end{gather*}

The $n$-simplex associated to the identity shuffle is referred to as the \textit{fundamental simplex} of $\simplex^{n_1} \times \dots \times \simplex^{n_k}$.
We refer to the simplicial map
\[
\incl \colon
\simplex^{n_1+\dots+n_k} \to
\simplex^{n_1} \times \dots \times \simplex^{n_k}
\]
defined by this simplex as the \textit{fundamental inclusion}, and to a left inverse of it
\[
\proj \colon
\simplex^{n_1} \times \dots \times \simplex^{n_k} \to
\simplex^{n_1+\dots+n_k}
\]
as a fundamental projection. \anibal{can we give a formula for $\pi$?}

\subsection{A finality lemma} \label{ss:finality}

Let $Y$ be a simplicial set.
Consider the functor
\[
\incl_{\downarrow Y} \colon (\simplex^{n_1} \times \dots \times \simplex^{n_k} \downarrow Y) \to (\simplex^{n_1+\dots+n_k} \downarrow Y)
\]
defined by the fundamental inclusion.
For any cocomplete category $\sC$ and functor $F \colon (\simplex^{n_1+\dots+n_k} \downarrow Y) \to \sC$ the natural morphism between colimits
\[
\colim_{(\simplex^{n_1} \times \dots \times \simplex^{n_k} \downarrow Y)} F \circ \incl_{\downarrow Y} \
\longrightarrow \
\colim_{(\simplex^{n_1+\dots+n_k} \downarrow Y)} F
\]
is an isomorphism.

\begin{proof}
	As explained for example in \cite[\subsectionSymbol8.3]{riehl2014categorical}, this is equivalent to category $(\incl_{\downarrow Y} \downarrow f)$ being non-empty and connected for every $f \colon \simplex^{n_1+\dots+n_k} \to Y$.
	These properties follow from $(\incl_{\downarrow Y} \downarrow f)$ having an initial object $f \circ \proj \colon \simplex^{n_1} \times \dots \times \simplex^{n_k} \to Y$.
\end{proof}

\subsection{Topological simplex and geometric realization}

We will use the following model of the topological simplex:
\[
\gsimplex^n = \big\{
(t_1, \dots, t_n) \in [0,1]^n \mid t_1 \geq \dots \geq t_n
\big\}
\]
with
\[
\delta_i(t_1, \dots, t_n) =
\begin{cases}
	(1, t_1, \dots, t_n) & i = 0, \\
	(t_1, \dots, t_i, t_i, \dots, t_n) & 0 < i < n, \\
	(t_1, \dots, t_n, 0) & i = n,
\end{cases}
\]
and
\[
\sigma_i(t_1, \dots, t_n) = (t_1, \dots, \widehat t_i, \dots, t_n).
\]
The \textit{geometric realization} functor
\[
\bars{-} \colon \mSet \to \Top
\]
is defined as the Yoneda extension of the functor defined on representable objects by
\[
\bars{\simplex^{n_1, \dots, n_k}} =
\gsimplex^{n_1} \times \dots \times \gsimplex^{n_k}
\]
Explicitly,
\begin{align*}
	\bars{X} &=
	\colim_{\cramped{(\simplex^{n_1, \dots, n_k} \downarrow X})} \, \bars{\simplex^{n_1, \dots, n_k}} \\ &=
	\coprod X_{n_1,\dots,n_k} \times \gsimplex^{n_1} \times \dots \times \gsimplex^{n_k} /_\sim
\end{align*}
where
\[
\begin{split}
(\dege_i^j(x), t_1, \dots, t_j, \dots, t_k) &\sim (x, t_1, \dots, \delta_i(t_j), \dots, t_k), \\
(\dege^j_i(x), t_1, \dots, t_j, \dots, t_k) &\sim (x, t_1, \dots, \sigma_i(t_j), \dots, t_k).
\end{split}
\]
As expected, the restriction of the geometric realization functor to the subcategory of $1$-fold multisimplicial sets recovers the usual geometric realization of simplicial sets up to natural equivalence.

\subsection{Eilenberg--Zilber subdivision}

We can think of $\bars{\simplex^{n_1} \times \dots \times \simplex^{n_k}}$ as a simplicial subdivision of $\gsimplex^{n_1} \times \dots \times \gsimplex^{n_k}$ as follows.
Let $n = n_1 + \dots + n_k$ and consider $(n_1, \dots, n_k)$-shuffles as automorphisms of $\set{1, \dots, n}$ instead of $\set{0, \dots, n-1}$.
The map
\[
\bars{\simplex^{n_1} \times \dots \times \simplex^{n_k}} =
\sh(n_1, \dots, n_k) \times \gsimplex^n /_\sim \ \longrightarrow \
\gsimplex^{n_1} \times \dots \times \gsimplex^{n_k}
\]
determined by $(\sigma, x_1, \dots, x_n) \mapsto (x_{\sigma(1)}, \dots, x_{\sigma(n)})$ is a well defined homeomorphism.
The inverse of this map
\[
\ez \colon
\gsimplex^{n_1} \times \dots \times \gsimplex^{n_k} \to
\bars{\simplex^{n_1} \times \dots \times \simplex^{n_k}}
\]
is cellular and we refer to it as the \textit{Eilenberg--Zilber subdivision}.

We will use this homeomorphism to identify $\gsimplex^{n_1} \times \dots \times \gsimplex^{n_k}$ and $\bars{\simplex^{n_1} \times \dots \times \simplex^{n_k}}$ as topological spaces.

Let $X$ be a multisimplicial set.
The natural composition
\begin{align*}
	\bars{X} \ &\xra{\cong}
	\colim_{\simplex^{n_1, \dots, n_k} \downarrow X} \ \bars{\simplex^{n_1, \dots, n_k}} \\ &\xra{=}
	\colim_{\simplex^{n_1, \dots, n_k} \downarrow X} \ \gsimplex^{n_1} \times \dots \times \gsimplex^{n_k} \\ &\xra{\ez}
	\colim_{\simplex^{n_1, \dots, n_k} \downarrow X} \ \bars{\simplex^{n_1} \times \dots \times \simplex^{n_k}} \\ &\xra{\cong}
	\bars{X^{\diag}}
\end{align*}
is a cellular map whose underlying continuous map is a homeomorphism.
We refer to this extension of $\ez$ also as \textit{Eilenberg--Zilber subdivision} and use the same notation for it.

\subsection{Cartan--Serre collapse} \label{ss:cartan-serre map}

The \textit{Cartan--Serre collapse} is the composition
\[
\cs \colon
\gsimplex^{n_1} \times \dots \times \gsimplex^{n_k} \to
[0,1]^{n_1 + \dots + n_k} \to
\gsimplex^{n_1 + \dots + n_k}
\]
where the first map is the canonical inclusion and the second is the projection
\[
(x_1, \dots, x_{n_1 + \dots + n_k}) \mapsto (x_1, x_1x_2, \dots, x_1x_2 \dots x_{n_1 + \dots + n_k}).
\]
We remark that the inclusion of the fundamental simplex
\[
\gsimplex^{n_1 + \dots + n_k} \to \bars{\simplex^{n_1} \times \dots \times \simplex^{n_k}}
\]
defines a cellular section of the Cartan--Serre collapse using the identification of $\gsimplex^{n_1} \times \dots \times \gsimplex^{n_k}$ and $\bars{\simplex^{n_1} \times \dots \times \simplex^{n_k}}$.

Let $Y$ be a simplicial set.
We have the following cellular homotopy equivalence also referred to as Cartan--Serre collapse and denoted by the same symbol:
\[
\begin{split}
\bars{\fM Y} & \xra{\cong}
\colim_{\simplex^{n_1, \dots, n_k} \downarrow \, \fM Y} \
\gsimplex^{n_1} \times \cdots \times \gsimplex^{n_k} \\ & \xra{\cong}
\colim_{\simplex^{n_1} \times \dots \times \simplex^{n_k} \downarrow \, Y} \
\gsimplex^{n_1} \times \cdots \times \gsimplex^{n_k} \\ & \xra{\cs}
\colim_{\simplex^{n_1} \times \dots \times \simplex^{n_k} \downarrow \, Y} \
\gsimplex^{n_1 + \dots + n_k} \\ & \xra{\cong}
\colim_{\simplex^{n} \downarrow \, Y} \,
\gsimplex^{n} \\ & \xra{\cong}
\bars{Y}
\end{split}
\]
\anibal{explain better. Missing use of adjoint}
where the third map is induced by regarding $\cs$ as a natural transformation and the fourth is a homeomorphism by \cref{ss:finality}.

\subsection{Quillen equivalence}

\anibal{Reference or prove that $(-)^{\diag} \dashv \fM$ forms a Quillen equivalence.}