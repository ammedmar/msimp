% !TEX root = ../msimplicial.tex

\section{Simplicial comparison}

We will use $\sSet$ to denote the category of $1$-fold multisimplicial sets $\mSet{1}$ referring to its objects as simplicial sets as usual.

\subsection{Diagonal simplicial set} \label{ss:diagonal simplicial set}

For any $k \in \N$, the diagonal
\[
\simplex^\op \xra{\diag}
(\simplex^\op)^{\times k} \xra{\cong}
(\simplex^{\times k})^{\op}
\]
induces a functor
\[
(-)^{\diag} \colon \mSet{k} \to \sSet
\]
explicitly defined on a $k$-fold multisimplicial set $X$ by
\begin{gather*}
	X^{\diag}_m = X_{m, \dots, m},
	\qquad
	\face_i = \face_i^1 \circ \dots \circ \face_i^k,
	\qquad
	\dege_i = \dege_i^1 \circ \dots \circ \dege_i^k.
\end{gather*}
It is straightforward to verify that
\[
\big(\simplex^{n_1,\dots,n_k}\big)^\diag \cong
\msimplex{n_1}{n_k}
\]
as simplicial sets.

The functor $(-)^\diag \colon \mSet{k} \to \sSet$ admits a right adjoint $\radj \colon \sSet \to \mSet{k}$, defined, as usual, by the expression
\[
\radj(Y)_{m_1,\dots,m_k} =
\sSet\big( \simplex^{m_1} \times \dots \times \simplex^{m_k}, \, Y \big).
\]
These functors define a Quillen equivalence.
A proof of this fact can be given \cite[Proposition~1.6.8]{maltsiniotis2005grothendieck}.

\subsection{Eilenberg--Zilber map}

Recall that an \textit{$(n_1,\dots,n_k)$-shuffle} $\sigma$ is a permutation in $\sym_n$ satisfying
\begin{gather*}
	\sigma(1) < \dots < \sigma(n_1), \\
	\sigma(n_1+1) < \dots < \sigma(n_1+n_2), \\
	\vdots \\
	\sigma(n-n_k-1) < \dots < \sigma(n),
\end{gather*}
where $n = n_1+\dots+n_k$.
We denote the set of such permutations by $\sh_{n_1,\dots,n_k}$.
For any $\sigma \in \sh_{n_1,\dots,n_k}$ the inclusion
\[
\gi_\sigma \colon \gsimplex^n \to \gmsimplex{n_1}{n_k}
\]
is defined by the assignment
\[
(x_1,\dots,x_n) \mapsto (x_{\sigma^{-1}(1)}, \dots, x_{\sigma^{-1}(n)}).
\]
If $e$ is the identity permutation, we denote $\gi_{e}$ simply as $\mathfrak{i}$.
The set $\set{\gi_\sigma \mid \sigma \in \sh_{n_1,\dots,n_k}}$ defines a triangulation of $\gmsimplex{n_1}{n_k}$ making it isomorphic, in the category cellular spaces, to the geometric realization of the simplicial set $\msimplex{n_1}{n_k}$.
Using this identification, the identity map induces a cellular map
\[
\ez \colon \gmsimplex{n_1}{n_k} \to \bars[\big]{\msimplex{n_1}{n_k}},
\]
whose induced chain map
\[
\EZ \colon \chains(\simplex^{n_1,\dots,n_k}) \to \chains\big(\msimplex{n_1}{n_k}\big),
\]
agrees, under the natural identifications, with the traditional Eilenberg--Zilber map.

For a multisimplicial set $X$, the induced chain map $\EZ \colon \chains(X) \to \chains(X^\diag)$ is explicitly given on an $(n_1,\dots,n_k)$-multisimplex $x$ by
\[
\EZ(x) \ = \sum_{\qquad \crampedclap{\sigma \in \sh_{n_1,\dots,n_k}}} \sign(\sigma) \, X(\sigma_1 ,\dots, \sigma_k)(x)
\]
%\[
%\EZ \big( [n_1] \times\dots\times [n_k] \big) =
%\sum_{\sigma \in \sh_{n_1,\dots,n_k}} \sign(\sigma) \, \sigma_1 \times\dots\times \sigma_k
%\]
where, for $\ell \in \set{1,\dots,k}$, the morphisms $\sigma_\ell \colon [n] \to [n_\ell]$ are defined by the following property: For
each $j \in \set{1,\dots,n}$ there is exactly one $\ell \in \set{1,\dots,k}$ such that $\sigma_\ell(j+1) = \sigma_\ell(j)+1$ and $\sigma_i(j+1) = \sigma_i(j)$ for all $i \neq \ell$.
\todo{@andrea: this is off, unfortunately. Could you please fix it?}

Since the traditional Eilenberg--Zilber map preserves counital coalgebra structures we have the following.

\begin{lemma}
	For every multisimplicial set $X$ the map $\EZ \colon \chains(X) \to \chains(X^\diag)$ is a quasi-isomorphism of counital coalgebras.
\end{lemma}

%\begin{proof}
%	Consider an $(n_1,\dots,n_k)$-multisimplex $x$ with characteristic map
%	\[
%	\zeta \colon \msimplex{n_1}{n_k} \to X.
%	\]
%	The statement follows from the fact that $\EZ(x) = \chains \zeta^\diag \circ \EZ \big([n_1] \ot\dots\ot [n_k]\big)$ and that the usual Eilenberg--Zilber map is a counital coalgebra morphisms.
%\end{proof}

We remark that the Eilenberg--Zilber map is not a morphisms of $E_\infty$-coalgebras.
For example, as shown in \cite[\S5.4]{medina2021cube_einfty}
\[
\Delta_1 \circ \EZ\big([0,1] \ot [0,1]\big) \neq
\EZ \, \circ \, \Delta_1\big([0,1] \ot [0,1]\big).
\]

\subsection{Canonical inclusion}

Let $Y$ be a simplicial set and $n$ an integer.
Consider the function $Y_n \to \cU Y_{n,0,\dots,0}$ sending a simplex with characteristic map $\zeta \colon \simplex^n \to Y$ to the composition
\[
\simplex^n \times \msimplex{0}{0} \xra{\pi_1} \simplex^n \xra{\zeta_y} Y.
\]
These functions induce a chain map
\[
\In \colon \chains(Y) \to \chains(\cU Y)
\]
and we have the following.

\begin{lemma*}
	The canonical inclusion $\In \colon \chains(Y) \to \chains(\cU Y)$ is a quasi-isomorphism of $E_\infty$-coalgebras for any simplicial set $Y$.
\end{lemma*}

\begin{proof}
	The structure preserving properties of this map are immediate.
	It remains to be shown that it induces a homology isomorphism.
	Consider the composition of quasi-isomorphisms
	\[
	\chains(\cU Y) \xra{\EZ}
	\chains\big((\cU Y)^\diag\big) \to
	\chains(Y)
	\]
	where the second map is induced by the counit of the adjunction.
	We will now verify that it is left inverse to $\In$.
	Consider a simplex $y$ with characteristic map $\zeta \colon \simplex^n \to Y$.
	The multisimplex $\In(y)$ is given by the simplicial map $\simplex^n \times \msimplex{0}{0} \xra{\pi_1} \simplex^n \xra{\zeta} Y$.
	Since the only $(n,0,\dots,0)$-shuffle is the identity, the simplex $\EZ \circ \In (y)$ is the simplicial map
	\[
	\zeta \circ \pi_1 \colon \simplex^n \times \msimplex{n}{n} \to Y.
	\]
	Finally, the image of this simplex under the counit is the evaluation of $[n] \times\dots\times [n]$ on $\zeta \circ \pi_1$ which gives $\zeta[n] = y$ as claimed.
\end{proof}

\subsection{Singular chains} \label{ss:singular}

\begin{lemma*}
	Let $\fZ$ be a topological space.
	The chain map
	\[
	\rS^{(1)}(\fZ) \to \rS^{(k)}(\fZ),
	\]
	defined by precomposing a continuous map $(\gsimplex^n \to \fZ)$ with the projection
	\[
	\gsimplex^n \times \gmsimplex{0}{0} \xra{\pi_1} \gsimplex^n,
	\]
	is a quasi-isomorphism of $E_\infty$-coalgebras.
\end{lemma*}

\begin{proof}
	This map factors as the composition of two quasi-isomorphism of $E_\infty$-coalgebras.
	The first is $\In \colon \rS^{(1)}(\fZ) \to \chains(\cU^{(k)} \Sing^{(1)}(\fZ))$, which was studied in the previous lemma.
	The second is induced from a multisimplicial isomorphism
	\[
	\cU^{(k)} \Sing^{(1)}(\fZ) \to \Sing^{(k)}(\fZ)
	\]
	defined as follows.
	Using the adjunction of \cref{ss:geometric realization}, any simplicial map $\msimplex{n_1}{n_k} \to \Sing^{(1)}(\fZ)$ corresponds canonically to a continuous map $\bars{\msimplex{n_1}{n_k}} \to \fZ$, which precomposing with $\ez$ gives a continuous map $\gmsimplex{n_1}{n_k} \to \fZ$.
	It is not hard to see that every such map arises this way since $\ez$ is a homeomorphism.
\end{proof}