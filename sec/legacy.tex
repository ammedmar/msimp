BARRATT PROPOSITION

%\begin{proposition}
%	The homomorphism $TC:\chi(k) \to BE(k)$ by Berger-Fresse satisfies
%	$$TC=N(tc) \circ \ez$$ where $\ez:N(\sur(k)) \to
%	N(\sur(k)^D)$ is the Eilenberg-Zilber map.
%\end{proposition}
%
%\begin{proof}
%	By inspection of the definition in \cite{BFsmall}.%more? signs? which convention?
%\end{proof}

%Also $\cW$ is filtered \cite{BFsmall}, i.e.
%there is a family of nested simplicial sets %$$\cW_1\sym_k \subset \cW_2 %\sym_k
% \subset \dots $$ such that
%$\cW\sym_k={\rm colim}_d %\cW_d\sym_k$, that form a filtration by %operads
%$$\cW_1 \subset \cW_2 \subset \dots %\cW$$

%The normalized chain functor defines
%$$BE_d(k):=N(\cW_d \sym_k)$$ so that the %operads $BE_d$
%form a filtration of $BE$.

Barratt Eccles filtration

\begin{definition}[Filtration of the Barratt--Eccles simplicial set $\cW\sym_k$]
	Fix an element $w\in (\cW\sym_k)_{d}$, $w=(w_{1},\dots , w_{d})$ with $w_{h}\in \sym_k$.
	For any pair $(i,j)$ with $i< j$, denote $w_{ij}=(w_{1,ij},\dots , w_{d,ij})$ where $w_{h,ij}$ is the subsequence of $w_{h}$ obtained omitting all the occurrences of elements different from $i$ and $j$.
	%	For example, let $(123,231,312) \in (\cW\sym_3)_{3}$, we have $(123,231,312)_{12}=(12,21,12)$, $(123,231,312)_{23}=(23,23,32)$ and $(123,231,312)_{13}=(13,31,31 )$.
	The element $w$ belongs to $\cW_{(\mu,\sigma)}\sym_k\subseteq \cW\sym_k$ if for all pairs $(i,j)$, $i< j$ either along the sequence $w_{ij}$ $i$ and $j$ swap less than $\mu_{ij}$ times,
	%	, that is, as before, the amount of times that the order of $i$ and $j$ change ,
	or they swap exactly $\mu_{ij}$ times and the first permutation $w_{1,ij}$ is equal to $\sigma_{ij}$.
	Notice that
	\begin{equation*}
		\label{def}
		\cW_{n}\sym_k=\bigcup_{\max_{i<j} (\mu_{ij})< n} \cW_{(\mu,\sigma)}\sym_{k}
	\end{equation*}
	\\
	\\
	%	Returning to the example above we have $\mu_{12}(123,231,312)=2$, $\mu_{23}(123,231,312)=1$ and $\mu_{13}(123,231,312)=1$ so that $(123,231,312)\in (\cW_{(\mu,\sigma)}\sym_3)_{3}$
	%	\begin{itemize}
		%		\item if $\mu_{12}>2$, $\mu_{23}>1$ and $\mu_{13}>1$ and any $\sigma\in \sym_{3}$;
		%		\item if some $\mu_{ij}=\mu_{ij}(w)$ and the first permutation of $w$ is equal to $\sigma_{ij}$.
		%	\end{itemize}
	%	In our example we obtain that, by equation \ref{def},
	%	$(123,231,312)\in (\cW_{3}\sym_3)_{3}$
	%	\\
	%	\\
	%	We have obtained the filtration $$\cW_1\sym_k \subset \cW_2 \sym_k
	%	\subset \cW_3 \sym_k
	%	\subset \dots $$
	%	that is the other one we need in $tc$ definition.
\end{definition}

% In both cases, to see even the role of permutations we'll use always pairs $(\mu,\sigma)$ with $\mu$ exactly the collection of variation numbers obtained from the initial element so that the permutation is uniquely determined, and $(\mu,\sigma)$ is minimal in the poset.
% \\
% In our examples
% \begin{itemize}
	% 	\item $31231\in \sur_{(\mu,\sigma)}(3)_{2,1,2} \subseteq \sur_{3}(3)_{2,1,2}$ with $(\mu_{12},\mu_{13},\mu_{23})=(1,2,1)$ and $\sigma=(312)$
	% 	\item $(123,231,312)\in (\cW_{(\mu,\sigma)}\sym_3)_{3} \subset (\cW_{3}\sym_3)_{3}$ with $(\mu_{12},\mu_{13},\mu_{23})=(2,1,1)$ and $\sigma=(123)$
	% \end{itemize}
%	We observe that the simplicial map $tc$ respects the filtration, sending
%	$\sur_d(k)^D$ to $\cW_d\sym_k$
%	\\
%	\begin{example}
	%	We are givine some example of the compatibility of $tc$ to visualize why it works in general.
	%	Let $122333112 \in \sur(3)^{D} $
	%	\begin{itemize}
		%		\item $(122333112)_{12}=122112\rightarrow \mu_{12}=2$
		%		\item $(122333112)_{13}=133311\rightarrow \mu_{13}=1$
		%		\item $(122333112)_{23}=223332\rightarrow \mu_{23}=1$
		%	\end{itemize}
	%So $\mu=(\mu_{12},\mu_{13},\mu_{23})=(2,1,1)$
	%and the first occurrences of $1,2,3$ give us the permutation $\sigma=(123)$ $\Rightarrow$ $122333112 \in \sur_{((2,1,1),(123))}(3)^{D}$ $\Rightarrow$ $122333112 \in \sur_{3}(3)^{D}$.
	%Now applying $tc$ we obtain $tc(122333112)=(123,231,312)\in (\cW\sym_3)_{3} $
	%\begin{itemize}
	%\item $(123,231,312)_{12}=(12,21,12)\rightarrow \mu_{12}=2$
	%\item $(123,231,312)_{13}=(13,31,31)\rightarrow \mu_{13}=1$
	%\item $(123,231,312)_{23}=(23,23,32)\rightarrow \mu_{23}=1$
	%\end{itemize}
	%So $\mu=(\mu_{12},\mu_{13},\mu_{23})=(2,1,1)$
	%and the first permutation in $(123,231,312)$ give us the permutation $\sigma=(123)$ $\Rightarrow$ $(123,231,312) \in (\cW_{((2,1,1),(123))}\sym_{3})_{3}$ $\Rightarrow$ $(123,231,312) \in (\cW_{3}\sym_{3})_{3}$.
	%
	%
	%
	%	\end{example}

OLD BARRATT-ECCLES

%The following is well known, see for example \cite{BF}
%\begin{proposition} \ref{be-real}
%	The geometric realization $\bars{\cW_d \sym_k}$ is $\sym_k$-equivariantly homotopy equivalent to $F_d(\R^k)$.
%\end{proposition}

%The (normalized) chain complex of $\cW_d\sym_k$ is called the Barratt--Eccles chain complex $$BE_d(k):=N(\cW_d\sym_k)$$
%the collection of these complexes forms the Barratt--Eccles operad $BE_d$.
%\cite{BF}
%There is a filtration
%$$BE_0 \subset BE_1 \subset BE_2 \dots \subset BE$$
%
%Similarly as for proposition \ref{sur-model} we have the following.
%
%\begin{proposition}
%	There is a  $\sym_k$-equivariant quasi-isomorphism
%	of $E_\infty$-coalgebras between $BE_d(k)$ and
%	$S(F_k(\R^d))$.
%\end{proposition}
%
%\begin{proof} \label{be-model}
%	The required quasi-isomorphism follows from proposition \ref{be-real} and the equivariant quasi-isomorphism $BE_d(k) \simeq S(|W_d\sym_k|)$. %counit of quillen adjunction
%\end{proof}


\section{Introduction old}

The normalized cochain complex $N^*(K)$ of a simplicial set $K$ is equipped with the classical Alexander-Whitney cup product that makes into a differential graded algebra.
Our first goal is to extend this product to the case of multisimplicial sets.
Let us consider a $k$-fold simplicial set $X$, that is a contravariant functor from $(\Delta)^k$ to the category of sets.
The restriction to the diagonal $\Delta \subset (\Delta)^k$ defines a simplicial set $X^D$.
There is a notion of geometric realization $|X|$ of a $k$-fold simplicial set $X$, that is a CW complex with a cell $e_x$ for each non-degenerate multisimplex $x$, with a characteristic map from a product of simplexes $$\Delta_{i_1} \times \dots \times \Delta_{i_k} \to e_x$$ This extends the classical case where the characteristic map has a single simplex as domain.
Quillen proved in \cite{Quillen} that there is a natural homeomorphism of realizations $|X| \cong |X^D|$.
Under this homeomorphism the cells of $|X^D|$ arise from those of $|X|$ by subdividing $k$-fold products of simplexes into simplexes.
This procedure is described combinatorially by the Eilenberg-Zilber quasi-isomorphism
$$EZ:N_*(X) \to N_*(X^D)$$

%that induces a quasi-isomorphism on normalized chains
%$N_*(X) \to N_*(X^D)$ after quotienting out degenerate %chains.
% As in the classical simplicial case, the projection $C_*(X) \to N_*(X)$ onto normalized chains
%is a quasi-isomorphism.

\medskip

We prove in Theorem \ref{algebra} %quale?
that the cochain complex $N^*(X)$ is equipped with a differential graded algebra structure.
The product is the natural extension to the multisimplicial case of the cup product defined by the Alexander-Whitney formula, by
evaluating cochains on front and rear faces in all multisimplicial directions.
%formula?
We prove in section \ref{ultima} that the dual Eilenberg-Zilber map
$$EZ^*:N^*(X^D) \to N^*(X)$$ %provato dove?
is a quasi-isomorphism of differential graded algebras, where the source is equipped with the classical cup product.
We extend the previous construction to an $E_\infty$ structure on multisimplicial cochains using the approach defined by the first author in \cite{anibal}.
In this case $EZ$ does not respect the $E_\infty$-structures, but there is a natural quasi-isomorphism in the opposite direction that does preserve them.
\paolo{anibal elaborates on Cartan-Serre?}

Our result is very useful for computations, since multisimplicial models of spaces have a significantly smaller number of non-degenerate cells then their simplicial models.
So $N^*(X)$ is much smaller than $N^*(X^D)$, but it contains the same information up to homotopy, allowing for example to calculate explicitly homology operations like Massey products, and the Steenrod algebra action.
As an example we consider a family of multisimplicial sets $Sur(k)$ defined by McClure and Smith, see \cite{MS}, modelling euclidean configuration spaces.
The proof by the third author of the non-formality of the cochain algebra of planar configuration spaces in \cite{formality} used the Barratt--Eccles simplicial model and the classical cup product.
Our new product on the multisimplicial McClure-Smith models makes the computation much simpler and faster, paving the way for an extension to higher dimensions.
%per esempio dire i numeri..


\subsection{Shuffles and the fundamental simplex} \label{ss:shuffles and fundamental simplex}

Let $n = n_1+\dots+n_k$.
An \textit{$(n_1,\dots,n_k)$-shuffle} $\sigma$ is an automorphism of $\set{0 + \dots + n-1}$ satisfying
\begin{gather*}
	\sigma(0) < \dots < \sigma(n_1-1), \\
	\sigma(n_1) < \dots < \sigma(n_1+n_2-1), \\
	\vdots \\
	\sigma(n_1+\dots+n_{k-1}) < \dots < \sigma(n_1+\dots+n_k-1).
\end{gather*}
The set of all of these -- which we denote $\sh(n_1,\dots,n_k)$ -- serves to parameterize the set of non-degenerate $n$-simplices of $\msimplex{n_1}{n_k}$ via the inclusion
\[
\begin{tikzcd}[column sep=tiny, row sep=0]
	\cI \colon &[-10pt]
	\sh(n_1,\dots,n_k) \arrow[r] &
	\big(\msimplex{n_1}{n_k}\big)_n \\ &
	\sigma \arrow[r, mapsto] &
	s_{V_1} [n_1] \times \dots \times s_{V_k} [n_k]
\end{tikzcd}
\]
where $s_{V_j} = s_{v_{n-n_j}^j} \dotsb \ s_{v_1^j}$ for $V_j = \set{v_1^j < \dots < v_{n-n_j}^j}$ and
\begin{gather*}
	V_1 = \set{0,\dots,n-1} \setminus \sigma^{-1} \set{0,\dots,n_1-1} \\
	V_2 = \set{0,\dots,n-1} \setminus \sigma^{-1} \set{n_1, \dots, n_1+n_2-1} \\
	\vdots \\
	V_k = \set{0,\dots,n-1} \setminus \sigma^{-1} \set{n_1+\dots+n_{k-1}, \dots, n_1+\dots+n_k-1}.
\end{gather*}

The $n$-simplex associated to the identity shuffle is referred to as the \textit{fundamental simplex} of $\msimplex{n_1}{n_k}$.
We refer to the simplicial map
\[
\incl \colon
\simplex^{n_1+\dots+n_k} \to
\msimplex{n_1}{n_k}
\]
defined by this simplex as the \textit{fundamental inclusion}, and to a left inverse of it
\[
\proj \colon
\msimplex{n_1}{n_k} \to
\simplex^{n_1+\dots+n_k}
\]
as a fundamental projection. \anibal{can we give a formula for $\pi$?}

\subsection{A finality lemma} \label{ss:finality}

Let $Y$ be a simplicial set.
Consider the functor
\[
\incl_{\downarrow Y} \colon (\msimplex{n_1}{n_k} \downarrow Y) \to (\simplex^{n_1+\dots+n_k} \downarrow Y)
\]
defined by the fundamental inclusion.
For any cocomplete category $\sC$ and functor $F \colon (\simplex^{n_1+\dots+n_k} \downarrow Y) \to \sC$ the natural morphism between colimits
\[
\colim_{(\msimplex{n_1}{n_k} \downarrow Y)} F \circ \incl_{\downarrow Y} \
\longrightarrow \
\colim_{(\simplex^{n_1+\dots+n_k} \downarrow Y)} F
\]
is an isomorphism.

\begin{proof}
	As explained for example in \cite[\subsectionSymbol8.3]{riehl2014categorical}, this is equivalent to category $(\incl_{\downarrow Y} \downarrow f)$ being non-empty and connected for every $f \colon \simplex^{n_1+\dots+n_k} \to Y$.
	These properties follow from $(\incl_{\downarrow Y} \downarrow f)$ having an initial object $f \circ \proj \colon \msimplex{n_1}{n_k} \to Y$.
\end{proof}


\newpage

One can use this description of $\ci$ to present simplicial maps $\ci_\sigma$ inducing the cellular maps $\gi_\sigma$ for any other shuffle $\sigma$, but we do not need those formulas.

Let $X$ be a multisimplicial set.
The natural composition
\begin{align*}
	\bars{X} \ &\xra{\cong}
	\colim_{\simplex^{n_1,\dots,n_k} \downarrow X} \ \bars{\simplex^{n_1,\dots,n_k}} \\ &\xra{=}
	\colim_{\simplex^{n_1,\dots,n_k} \downarrow X} \ \gmsimplex{n_1}{n_k} \\ &\xra{\ez}
	\colim_{\simplex^{n_1,\dots,n_k} \downarrow X} \ \bars{\msimplex{n_1}{n_k}} \\ &\xra{\cong}
	\bars{X^{\diag}}
\end{align*}
is a cellular map whose underlying continuous map is a homeomorphism.
We refer to this extension of $\ez$ also as \textit{Eilenberg--Zilber subdivision} and use the same notation for it.

\subsection{Cartan--Serre map} \label{ss:cartan-serre map}

The \textit{Cartan--Serre map} is the composition
\[
\cs \colon
\bars{\msimplex{n_1}{n_k}} \to
[0,1]^{n_1+\dots+n_k} \to
\gsimplex^{n_1+\dots+n_k}
\]
where the first map is the canonical inclusion and the second is the projection
\[
(x_1,\dots,x_{n_1+\dots+n_k}) \mapsto
(x_1, x_1x_2, \dots, x_1x_2 \dots x_{n_1+\dots+n_k}).
\]
It is induced by the geometric realization of following composition of simplicial maps
\[
\ccs \colon
\msimplex{n_1}{n_k} \xra{\ci\! \times\dots\times \ci}
\scube{n_1} \times\dots\times \scube{n_k} \xra{\cong}
\scube{n} \to \simplex^n
\]
where the last one is defined for every $m \in \N$ by
\[
[\varepsilon_0^1, \dots, \varepsilon_m^1]
\times \dots \times
[\varepsilon_0^n, \dots, \varepsilon_m^n]
\mapsto
[v_0, \dots, v_m],
\]
with each $\varepsilon_i^j \in \set{0,1}$ and such that
\[
v_i \defeq \varepsilon_i^1 + \varepsilon_i^1 \varepsilon_i^2 + \dots + \varepsilon_i^1 \dotsm \varepsilon_i^n.
\]
\begin{lemma*}
	If either $j \in \set{1,\dots,k-1}$ and $i \in \set{0,\dots,n_j-1}$ or $j = k$, then
	\[
	\begin{tikzcd}
			\msimplex{n_1}{n_k} \arrow[d,"\coface_i^j"'] \arrow[r,"\ccs"] &
			\simplex^n \arrow[d,"\coface_{n_1+\dots+n_{j-1}+i}"] \\
			\simplex^{n_1} \times\dots\times \simplex^{n_j+1} \times\dots\times \simplex^{n_k} \arrow[r,"\ccs"] &
			\simplex^{n+1}
		\end{tikzcd}
	\]
	commutes.
%	In particular, in this case we have
%	\[
%	\CS \circ \chains(\coface_i^j) =
%	\chains(\coface_{n_1+\dots+n_{j-1}+i}) \circ \CS.
%	\]
%	Additionally, in other cases $\coface^j_i$ is degenerate.
\end{lemma*}

This lemma ensures that the following natural map is well defined for every simplicial set $Y$.
Let
\[
\cs_Y \colon \bars{Y} \to \bars{\radj Y}
\]
be defined by the assignment
\[
\gsimplex^n \times Y_n \ni (x, y) \mapsto (x, \xi_y \circ \ccs) \in \gmsimplex{n_1}{n_k} \times \sSet()
\]

\begin{remark*}
	We mention, although we do not use, that the following diagram commutes up to a cellular homotopy
	\[
	\begin{tikzcd}
			\gmsimplex{n_1}{n_k} \arrow[d,"\coface_i^j"'] \arrow[r,"\cs"] &
			\gsimplex^n \arrow[d,"\coface_{n_1+\dots+n_{j-1}+i}"] \\
			\gsimplex^{n_1} \times\dots\times \gsimplex^{n_i+1} \times\dots\times \gsimplex^{n_k} \arrow[r,"\cs"] &
			\gsimplex^{n+1}
		\end{tikzcd}
	\]
	if either $j \in \set{1,\dots,k-1}$ and $i \in \set{0,\dots,n_j-1}$ or $j = k$.
	In particular, in this case we have
	\[
	\CS \circ \chains(\coface_i^j) =
	\chains(\coface_{n_1+\dots+n_{j-1}+i}) \circ \CS.
	\]
	Additionally, in other cases $\coface^j_i$ is degenerate.
\end{remark*}

\begin{remark*}
	Please notice that the (cellular) inclusion
	\[
	\gi \colon \gsimplex^{n_1+\dots+n_k} \to
	\bars{\msimplex{n_1}{n_k}}
	\]
	is a section of the Cartan--Serre collapse using the identification of $\gmsimplex{n_1}{n_k}$ and $\bars{\msimplex{n_1}{n_k}}$.
\end{remark*}

Let $Y$ be a simplicial set.
We have the following cellular homotopy equivalence also referred to as Cartan--Serre collapse and denoted by the same symbol:
\[
\begin{split}
\bars{\fM Y} & \xra{\cong}
\colim_{\simplex^{n_1,\dots,n_k} \downarrow \, \fM Y} \
\gsimplex^{n_1} \times\dots\times \gsimplex^{n_k} \\ & \xra{\cong}
\colim_{\msimplex{n_1}{n_k} \downarrow \, Y} \
\gsimplex^{n_1} \times\dots\times \gsimplex^{n_k} \\ & \xra{\cs}
\colim_{\msimplex{n_1}{n_k} \downarrow \, Y} \
\gsimplex^{n_1+\dots+n_k} \\ & \xra{\cong}
\colim_{\simplex^{n} \downarrow \, Y} \,
\gsimplex^{n} \\ & \xra{\cong}
\bars{Y}
\end{split}
\]
\anibal{explain better. Missing use of adjoint}
where the third map is induced by regarding $\cs$ as a natural transformation and the fourth is a homeomorphism by \cref{ss:finality}.


\subsection{EZ}

under the identification $\chains(\simplex^{n_1,\dots,n_k}) \cong \chains(\simplex^{n_1}) \ot \dotsb \ot \chains(\simplex^{n_k})$

The composition
\[
\gsimplex^n \xra{\gi}
\gmsimplex{n_1}{n_k} \xra{\ez}
\bars[\big]{\msimplex{n_1}{n_k}}
\]
is the geometric realization of the simplicial map
\[
\ci \colon \simplex^n \to \msimplex{n_1}{n_k}
\]
defined by the assignment
\[
[n] \mapsto s^{(n \setminus n_1)} [n_1] \times\dots\times s^{(n \setminus n_k)} [n_k]
\]
where
\begin{multline*}
	s^{(n \setminus n_i)} \defeq
	s_{n} \circ\dotsm\circ s_{n_1+\dots+n_i} \\
	\circ \widehat{s}_{n_1+\dots+n_i-1} \circ\dotsm\circ \widehat{s}_{n_1+\dots+n_{i-1}} \\ \circ
	s_{n_1+\dots+n_{i-1}-1}  \circ\dotsm\circ s_0.
\end{multline*}
\begin{equation*}
	s^{(n \setminus n_i)} \defeq
	s_{n} \circ\dotsm\circ s_{n_1+\dots+n_i} \circ s_{n_1+\dots+n_{i-1}-1}  \circ\dotsm\circ s_0.
\end{equation*}
An important example is given by the inclusion into the \textit{simplicial cube}
\[
\ci \colon \simplex^n \to \scube{n},
\]
explicitly given by
\[
[0,\dots,n] \mapsto
[0,1,1,\dots,1] \times [0,0,1,\dots,1] \times\dots\times [0,0,\dots,0,1].
\]