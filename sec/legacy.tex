\subsection{Shuffles and the fundamental simplex} \label{ss:shuffles and fundamental simplex}

Let $n = n_1+\dots+n_k$.
An \textit{$(n_1,\dots,n_k)$-shuffle} $\sigma$ is an automorphism of $\set{0 + \dots + n-1}$ satisfying
\begin{gather*}
	\sigma(0) < \dots < \sigma(n_1-1), \\
	\sigma(n_1) < \dots < \sigma(n_1+n_2-1), \\
	\vdots \\
	\sigma(n_1+\dots+n_{k-1}) < \dots < \sigma(n_1+\dots+n_k-1).
\end{gather*}
The set of all of these -- which we denote $\sh(n_1,\dots,n_k)$ -- serves to parameterize the set of non-degenerate $n$-simplices of $\msimplex{n_1}{n_k}$ via the inclusion
\[
\begin{tikzcd}[column sep=tiny, row sep=0]
	\cI \colon &[-10pt]
	\sh(n_1,\dots,n_k) \arrow[r] &
	\big(\msimplex{n_1}{n_k}\big)_n \\ &
	\sigma \arrow[r, mapsto] &
	s_{V_1} [n_1] \times \dots \times s_{V_k} [n_k]
\end{tikzcd}
\]
where $s_{V_j} = s_{v_{n-n_j}^j} \dotsb \ s_{v_1^j}$ for $V_j = \set{v_1^j < \dots < v_{n-n_j}^j}$ and
\begin{gather*}
	V_1 = \set{0,\dots,n-1} \setminus \sigma^{-1} \set{0,\dots,n_1-1} \\
	V_2 = \set{0,\dots,n-1} \setminus \sigma^{-1} \set{n_1, \dots, n_1+n_2-1} \\
	\vdots \\
	V_k = \set{0,\dots,n-1} \setminus \sigma^{-1} \set{n_1+\dots+n_{k-1}, \dots, n_1+\dots+n_k-1}.
\end{gather*}

The $n$-simplex associated to the identity shuffle is referred to as the \textit{fundamental simplex} of $\msimplex{n_1}{n_k}$.
We refer to the simplicial map
\[
\incl \colon
\simplex^{n_1+\dots+n_k} \to
\msimplex{n_1}{n_k}
\]
defined by this simplex as the \textit{fundamental inclusion}, and to a left inverse of it
\[
\proj \colon
\msimplex{n_1}{n_k} \to
\simplex^{n_1+\dots+n_k}
\]
as a fundamental projection. \anibal{can we give a formula for $\pi$?}

\subsection{A finality lemma} \label{ss:finality}

Let $Y$ be a simplicial set.
Consider the functor
\[
\incl_{\downarrow Y} \colon (\msimplex{n_1}{n_k} \downarrow Y) \to (\simplex^{n_1+\dots+n_k} \downarrow Y)
\]
defined by the fundamental inclusion.
For any cocomplete category $\sC$ and functor $F \colon (\simplex^{n_1+\dots+n_k} \downarrow Y) \to \sC$ the natural morphism between colimits
\[
\colim_{(\msimplex{n_1}{n_k} \downarrow Y)} F \circ \incl_{\downarrow Y} \
\longrightarrow \
\colim_{(\simplex^{n_1+\dots+n_k} \downarrow Y)} F
\]
is an isomorphism.

\begin{proof}
	As explained for example in \cite[\subsectionSymbol8.3]{riehl2014categorical}, this is equivalent to category $(\incl_{\downarrow Y} \downarrow f)$ being non-empty and connected for every $f \colon \simplex^{n_1+\dots+n_k} \to Y$.
	These properties follow from $(\incl_{\downarrow Y} \downarrow f)$ having an initial object $f \circ \proj \colon \msimplex{n_1}{n_k} \to Y$.
\end{proof}


\newpage

One can use this description of $\ci$ to present simplicial maps $\ci_\sigma$ inducing the cellular maps $\gi_\sigma$ for any other shuffle $\sigma$, but we do not need those formulas.

Let $X$ be a multisimplicial set.
The natural composition
\begin{align*}
	\bars{X} \ &\xra{\cong}
	\colim_{\simplex^{n_1,\dots,n_k} \downarrow X} \ \bars{\simplex^{n_1,\dots,n_k}} \\ &\xra{=}
	\colim_{\simplex^{n_1,\dots,n_k} \downarrow X} \ \gmsimplex{n_1}{n_k} \\ &\xra{\ez}
	\colim_{\simplex^{n_1,\dots,n_k} \downarrow X} \ \bars{\msimplex{n_1}{n_k}} \\ &\xra{\cong}
	\bars{X^{\diag}}
\end{align*}
is a cellular map whose underlying continuous map is a homeomorphism.
We refer to this extension of $\ez$ also as \textit{Eilenberg--Zilber subdivision} and use the same notation for it.

\subsection{Cartan--Serre map} \label{ss:cartan-serre map}

The \textit{Cartan--Serre map} is the composition
\[
\cs \colon
\bars{\msimplex{n_1}{n_k}} \to
[0,1]^{n_1+\dots+n_k} \to
\gsimplex^{n_1+\dots+n_k}
\]
where the first map is the canonical inclusion and the second is the projection
\[
(x_1,\dots,x_{n_1+\dots+n_k}) \mapsto
(x_1, x_1x_2, \dots, x_1x_2 \dots x_{n_1+\dots+n_k}).
\]
It is induced by the geometric realization of following composition of simplicial maps
\[
\ccs \colon
\msimplex{n_1}{n_k} \xra{\ci\! \times\dots\times \ci}
\scube{n_1} \times\dots\times \scube{n_k} \xra{\cong}
\scube{n} \to \simplex^n
\]
where the last one is defined for every $m \in \N$ by
\[
[\varepsilon_0^1, \dots, \varepsilon_m^1]
\times \dots \times
[\varepsilon_0^n, \dots, \varepsilon_m^n]
\mapsto
[v_0, \dots, v_m],
\]
with each $\varepsilon_i^j \in \set{0,1}$ and such that
\[
v_i \defeq \varepsilon_i^1 + \varepsilon_i^1 \varepsilon_i^2 + \dots + \varepsilon_i^1 \dotsm \varepsilon_i^n.
\]
\begin{lemma*}
	If either $j \in \set{1,\dots,k-1}$ and $i \in \set{0,\dots,n_j-1}$ or $j = k$, then
	\[
	\begin{tikzcd}
			\msimplex{n_1}{n_k} \arrow[d,"\coface_i^j"'] \arrow[r,"\ccs"] &
			\simplex^n \arrow[d,"\coface_{n_1+\dots+n_{j-1}+i}"] \\
			\simplex^{n_1} \times\dots\times \simplex^{n_j+1} \times\dots\times \simplex^{n_k} \arrow[r,"\ccs"] &
			\simplex^{n+1}
		\end{tikzcd}
	\]
	commutes.
%	In particular, in this case we have
%	\[
%	\CS \circ \chains(\coface_i^j) =
%	\chains(\coface_{n_1+\dots+n_{j-1}+i}) \circ \CS.
%	\]
%	Additionally, in other cases $\coface^j_i$ is degenerate.
\end{lemma*}

This lemma ensures that the following natural map is well defined for every simplicial set $Y$.
Let
\[
\cs_Y \colon \bars{Y} \to \bars{\radj Y}
\]
be defined by the assignment
\[
\gsimplex^n \times Y_n \ni (x, y) \mapsto (x, \xi_y \circ \ccs) \in \gmsimplex{n_1}{n_k} \times \sSet()
\]

\begin{remark*}
	We mention, although we do not use, that the following diagram commutes up to a cellular homotopy
	\[
	\begin{tikzcd}
			\gmsimplex{n_1}{n_k} \arrow[d,"\coface_i^j"'] \arrow[r,"\cs"] &
			\gsimplex^n \arrow[d,"\coface_{n_1+\dots+n_{j-1}+i}"] \\
			\gsimplex^{n_1} \times\dots\times \gsimplex^{n_i+1} \times\dots\times \gsimplex^{n_k} \arrow[r,"\cs"] &
			\gsimplex^{n+1}
		\end{tikzcd}
	\]
	if either $j \in \set{1,\dots,k-1}$ and $i \in \set{0,\dots,n_j-1}$ or $j = k$.
	In particular, in this case we have
	\[
	\CS \circ \chains(\coface_i^j) =
	\chains(\coface_{n_1+\dots+n_{j-1}+i}) \circ \CS.
	\]
	Additionally, in other cases $\coface^j_i$ is degenerate.
\end{remark*}

\begin{remark*}
	Please notice that the (cellular) inclusion
	\[
	\gi \colon \gsimplex^{n_1+\dots+n_k} \to
	\bars{\msimplex{n_1}{n_k}}
	\]
	is a section of the Cartan--Serre collapse using the identification of $\gmsimplex{n_1}{n_k}$ and $\bars{\msimplex{n_1}{n_k}}$.
\end{remark*}

Let $Y$ be a simplicial set.
We have the following cellular homotopy equivalence also referred to as Cartan--Serre collapse and denoted by the same symbol:
\[
\begin{split}
\bars{\fM Y} & \xra{\cong}
\colim_{\simplex^{n_1,\dots,n_k} \downarrow \, \fM Y} \
\gsimplex^{n_1} \times\dots\times \gsimplex^{n_k} \\ & \xra{\cong}
\colim_{\msimplex{n_1}{n_k} \downarrow \, Y} \
\gsimplex^{n_1} \times\dots\times \gsimplex^{n_k} \\ & \xra{\cs}
\colim_{\msimplex{n_1}{n_k} \downarrow \, Y} \
\gsimplex^{n_1+\dots+n_k} \\ & \xra{\cong}
\colim_{\simplex^{n} \downarrow \, Y} \,
\gsimplex^{n} \\ & \xra{\cong}
\bars{Y}
\end{split}
\]
\anibal{explain better. Missing use of adjoint}
where the third map is induced by regarding $\cs$ as a natural transformation and the fourth is a homeomorphism by \cref{ss:finality}.


\subsection{EZ}

under the identification $\chains(\simplex^{n_1,\dots,n_k}) \cong \chains(\simplex^{n_1}) \ot \dotsb \ot \chains(\simplex^{n_k})$

The composition
\[
\gsimplex^n \xra{\gi}
\gmsimplex{n_1}{n_k} \xra{\ez}
\bars[\big]{\msimplex{n_1}{n_k}}
\]
is the geometric realization of the simplicial map
\[
\ci \colon \simplex^n \to \msimplex{n_1}{n_k}
\]
defined by the assignment
\[
[n] \mapsto s^{(n \setminus n_1)} [n_1] \times\dots\times s^{(n \setminus n_k)} [n_k]
\]
where
\begin{multline*}
	s^{(n \setminus n_i)} \defeq
	s_{n} \circ\dotsm\circ s_{n_1+\dots+n_i} \\
	\circ \widehat{s}_{n_1+\dots+n_i-1} \circ\dotsm\circ \widehat{s}_{n_1+\dots+n_{i-1}} \\ \circ
	s_{n_1+\dots+n_{i-1}-1}  \circ\dotsm\circ s_0.
\end{multline*}
\begin{equation*}
	s^{(n \setminus n_i)} \defeq
	s_{n} \circ\dotsm\circ s_{n_1+\dots+n_i} \circ s_{n_1+\dots+n_{i-1}-1}  \circ\dotsm\circ s_0.
\end{equation*}
An important example is given by the inclusion into the \textit{simplicial cube}
\[
\ci \colon \simplex^n \to \scube{n},
\]
explicitly given by
\[
[0,\dots,n] \mapsto
[0,1,1,\dots,1] \times [0,0,1,\dots,1] \times\dots\times [0,0,\dots,0,1].
\]