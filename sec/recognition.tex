% !TEX root = ../msimp.tex

\subsection{Recognition of configuration spaces}\label{ss:recognition}

Let $\conf{r}{d}$ denote the configuration space of $r$-tuples of pairwise disjoint vectors in $\R^{d}$.
This space is equipped with a free action of the symmetric group $\sym_r$ of permutations of $\{1,\dots,r\}$ swapping elements of a $r$-tuple.

\begin{definition}
	A \textit{complete graph} on $r$ vertices is a pair $(\mu,\sigma)$ with $\mu$ a collection of non-negative integers $\mu_{ij}$ for all $1 \leq i < j \leq r$, and $\sigma$ is an ordering of $\{1,\dots,r\}$.
	We write $\sigma_{ij}$ for the restriction of the ordering $\sigma$ to the set $\{i,j\}$.
	Graphically $(\mu,\sigma)$ is a simple directed graph in the edge corresponding to $i<j$ directed according to $\sigma_{ij}$ and labeled by $\mu_{ij}$.
	Please consult \cref{f:complete graph} for an example.
	Let us denote the set of complete graphs with $r$ vertices by $\CG(r)$ equipped with the poset structure
	\begin{equation*}
		(\mu,\sigma)\le (\nu,\tau) \ \ \iff \ \
		\forall i,j \ (\mu_{ij}<\nu_{ij}) \ \text{ or } \
		(\mu_{ij},\sigma_{ij})= (\nu_{ij},\tau_{ij})
	\end{equation*}
	for each pair $i<j$.
	It is equipped with an exhaustive filtration by subposets
	\[
	\CG_1(r) \subset \CG_2(r) \subset \dotsb
	\]
	where $\CG_d(r)$ consists of those graphs with $\max(\mu_{ij}) < d$.
\end{definition}

\begin{figure}
	\centering
	\begin{equation*}
		\begin{tikzcd}
			\ & 2 & \ \\
			1 \arrow[ur,"2"] \arrow[rr] \arrow[dr,"3"']& \arrow[r,"1"] \arrow[u,"2"] & 3 \arrow[ul,"1"'] \\
			\ & 4 \arrow[ur,"4"'] \arrow[uu] & \
		\end{tikzcd}
	\end{equation*}
	\caption{A complete graph on 4 vertices with ordering $\sigma=(1432)$ and $\mu=(\mu_{12},\mu_{13},\mu_{14},\mu_{23},\mu_{24},\mu_{34})=(2,1,3,1,2,4)$.}
	\label{f:complete graph}
\end{figure}

\begin{definition}\label{cellulardecomposition}
	For a given poset $A$, a cellular $A$-decomposition of a topological space $\fX$ is a family of subspaces $\set{\fX_a}_{a \in A}$ such that:
	\begin{itemize}
		\item [i.] $a \leq b$ implies $\fX_a \subseteq \fX_b$;
		\item [ii.] $\colim_{a \in A} \fX_a = \fX$;
		\item [iii.] $\fX_a$ is contractible for each $a$;
		\item [iv.] $\bigcup_{a<b} \fX_a \subset \fX_b$ is a closed cofibration.
	\end{itemize}
\end{definition}

The relevance of this notion is the well-known fact that if a topological space $\fX$ admits a \textit{cellular $A$-decomposition}, then the natural maps
\begin{equation}\label{e:cellular poset decomposition}
	\fX = \colim_A \fX_a \leftarrow \hocolim_A \fX_a \to \bars{A}
\end{equation}
are cellular homotopy equivalences.
Please consult \cite[\S1.7]{berger1997confspacemodel} for a proof.

Let $\cC_d(r)$ be the space of $r$ little $d$-dimensional cubes, which is equipped with an equivariant homotopy equivalence to $\conf{r}{d}$ picking the center of cubes.
Brun and others in \cite{brunfiedorowiczvogt2007} show that $\cC_d(r)$ has a cellular $\CG^\ex_d(r)$-decomposition $\{\cC_a\}$, where $\CG^\ex_d(r)$ is a poset containing the poset $\CG_d(r)$ and the inclusion of posets induces an equivariant homotopy equivalence on realizations.
Combining these results we have

\begin{proposition}\label{p:zig-zag conf}
	If a space $\fX$ has a cellular $\CG_d(r)$-decomposition, then $\fX$ is equal to $\colim_{\CG_d(r)} \fX_a$ and
	\begin{equation*}
		\begin{tikzcd}[column sep=small, row sep=5]
			\displaystyle \colim_{\CG_d(r)} \fX_a & \arrow[l] \displaystyle \hocolim_{\CG_d(r)} \fX_a \arrow[r] & \bars{\CG_d(r)} \arrow[d] & & \\ & &
			\bars{\CG^\ex_d(r)} & \arrow[l] \displaystyle \hocolim_{\CG^\ex_d(r)}\,\cC_\alpha \arrow[r] & \cC_d(r) \arrow[r] & \conf{r}{d}
		\end{tikzcd}
	\end{equation*}
	is a zig-zag of equivariant homotopy equivalences.
\end{proposition}

\begin{definition}
	Let $X$ be a multisimplicial (or simplicial) set.
	A \textit{$\CG(r)$-filtration} of $X$ is a family of (multi)simplicial subsets $\{X_a\}$ indexed by
	$a \in \CG(r)$ so that
	\begin{enumerate}
		\item $a \leq b$ implies $X_a \subseteq X_b$;
		\item $|X_a|$ is a cellular $\CG(r)$-decomposition of the realization $|X|$
	\end{enumerate}
	In particular this implies that $X=\colim_{a \in \CG(r)}X_a$.
	Let $X_d = \colim_{a \in CG_d(r)} X_a$.
	There is a nested sequence
	\[
	X_1 \subset X_2 \subset \dotsb
	\]
	For a given (multi)simplex $x \in X$ we will refer to $\min\set{d \mid x \in X_d}$ as the \textit{complexity} of $x$.
\end{definition}