\subsection{Recognition of configuration spaces}\label{ss:recognition}



Let $\con{r}(\R^{d})$ denote the configuration space of $r$-tuples of pairwise disjoint vectors in $\R^{d}$.
This space is equipped with a free action of the symmetric group $\Sigma_r$ of permutations of $\{1,\dots,r\}$ swapping elements of a $r$-tuple.

%In this section we refine the filtrations of McClure-Smith and Barratt--Eccles in order to show that $tc$ induces equivariant weak equivalence on the filtation terms.

\begin{definition}[Complete graphs]
	A \textit{complete graph} on $r$ vertices is a pair $(\mu,\sigma)$, where $\mu$ is a collection of non-negative integers $\mu_{ij}\in\mathbb{N}$, $1 \leq i < j \leq r$
	and $\sigma$ is an ordering of
	$\{1,\dots,r\}$. Graphically $(\mu,\sigma)$ is a simple directed graph in which every couple of vertices $i,j$ is labeled with $\mu_{ij}$ and oriented consistently with the ordering.
	Let us denote the set of complete graphs with $r$ vertices as $\mathcal{CG}(r)$ and with $\mathcal{CG}_{n}(r)$ the collection of those graphs with $\max_{i<j}(\mu_{ij})< n$. We write $\sigma_{ij}$ for the restriction of the ordering $\sigma$ to the set $\{i,j\}$.
\end{definition}
	For example, the following figure represents a complete graph on 4 vertices with ordering $\sigma=(1432)$ and $\mu=(\mu_{12},\mu_{13},\mu_{14},\mu_{23},\mu_{24},\mu_{34})=(2,1,3,1,2,4)$.
	\begin{equation*}
		\begin{tikzcd}
			\ & 2 & \  \\
			1 \arrow[ur,"2"] \arrow[rr] \arrow[dr,"3"']& \arrow[r,"1"] \arrow[u,"2"] & 3 \arrow[ul,"1"']  \\
			\ & 4 \arrow[ur,"4"'] \arrow[uu] & \
		\end{tikzcd}
	\end{equation*}
	

\begin{definition}[Complete graphs form a poset]
	$\mathcal{CG}(k)$ has a poset structure as follow:
	\begin{equation*}
		(\mu,\sigma)\le (\nu,\tau) \ \ \text{ if and only if } \ \ (\mu_{ij}<\nu_{ij}) \ \ \text{ or } \ \ (\mu_{ij},\sigma_{ij})= (\nu_{ij},\tau_{ij})
	\end{equation*}
	for each pair $\left\lbrace i,j\right\rbrace \subset\left\lbrace 1,\dots,k  \right\rbrace $. This poset structure descends naturally to $\mathcal{CG}_{n}(k)$.
\end{definition}
We recall a definition due to Berger \cite{berger1997confspacemodel}, compare also \cite{beuckelmann2021master}.
\begin{definition}
For a given poset $A$, a cellular $A$-decomposition of a topological space $X$ is a family of subspaces $X_a \subseteq X$ indexed by $a \in A$ such that
\begin{itemize}
    \item if $a \leq b$ then $X_a \subseteq X_b$;
\item $\bigcup_{a \in A} X_a = X$;
\item $X_a$ is contractible for each $a$;
\item the inclusion $\bigcup_{a<b} X_a \subset X_b$ is a closed cofibration.
\end{itemize}
\end{definition} 

\begin{proposition}
If a topological space $X$ admits a \textit{cellular $A$-decomposition} induced by a poset $A$, then there is a homotopy equivalence between $X$ and the geometric realization of $A$. 
\end{proposition}

The case of interest for us is the following.

\begin{theorem} (Berger,\cite[\S1.13]{berger1997confspacemodel})
If a space has a cellular $\mathcal{CG}_d(r)$-decomposition then it is homotopy equivalent to 
the configuration space 
$\con{r}(\R^{d})$.
\end{theorem}

We warn the reader that configuration spaces themselves do not admit a cellular decomposition with respect to graph complete posets, but with respect to some extended versions of these posets, as explained in \cite{beuckelmann2021master}.

%Unfortunately, Euclidean configuration spaces do not admit a cellular $\mathcal{CG}$-decomposition. However, it is still possible to give a combinatorial model for these spaces using Smith-filtration from \cite{smith1989filtration}, in the context of simplicial sets, or equivalently using Surjection-filtration, in the context of multisimplicial sets.



