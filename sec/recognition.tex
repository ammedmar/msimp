\subsection{Recognition of configuration spaces}\label{ss:recognition}

Let $\con{r}(\R^{d})$ denote the configuration space of $r$-tuples of pairwise disjoint vectors in $\R^{d}$.
This space is equipped with a free action of the symmetric group $\sym_r$ of permutations of $\{1,\dots,r\}$ swapping elements of a $r$-tuple.

\begin{definition}[Complete graph]
	A \textit{complete graph} on $r$ vertices is a pair $(\mu,\sigma)$, where $\mu$ is a collection of non-negative integers $\mu_{ij}\in\mathbb{N}$, $1 \leq i < j \leq r$
	and $\sigma$ is an ordering of $\{1,\dots,r\}$.
	We write $\sigma_{ij}$ for the restriction of the ordering $\sigma$ to the set $\{i,j\}$.
	Graphically $(\mu,\sigma)$ is a simple directed graph in which every couple of vertices $i,j$ is labeled with $\mu_{ij}$ and oriented consistently with $\sigma_{ij}$.
	Let us denote the set of complete graphs with $r$ vertices by $\CG(r)$ equipped with the poset structure
	\begin{equation*}
		(\mu,\sigma)\le (\nu,\tau) \ \ \text{ if and only if } \ \
		(\mu_{ij}<\nu_{ij}) \ \ \text{ or } \ \
		(\mu_{ij},\sigma_{ij})= (\nu_{ij},\tau_{ij})
	\end{equation*}
	for each pair $i<j$.
	It is equipped with a filtration by subposets
	\[
	\CG_1(r) \subset \CG_2(r) \subset \dotsb
	\]
	where $\CG_n(r)$ consists of those graphs with $\max(\mu_{ij})< n$.
\end{definition}

\begin{figure}
	\centering
	\begin{equation*}
		\begin{tikzcd}
			\ & 2 & \  \\
			1 \arrow[ur,"2"] \arrow[rr] \arrow[dr,"3"']& \arrow[r,"1"] \arrow[u,"2"] & 3 \arrow[ul,"1"']  \\
			\ & 4 \arrow[ur,"4"'] \arrow[uu] & \
		\end{tikzcd}
	\end{equation*}
	\caption{A complete graph on 4 vertices with ordering $\sigma=(1432)$ and $\mu=(\mu_{12},\mu_{13},\mu_{14},\mu_{23},\mu_{24},\mu_{34})=(2,1,3,1,2,4)$.}
	\label{f:complete graph}
\end{figure}

\begin{definition}
	For a given poset $A$, a cellular $A$-decomposition of a topological space $X$ is a family of subspaces $X_a \subseteq X$ indexed by $a \in A$ such that:
	\begin{itemize}
		\item $a \leq b$ implies $X_a \subseteq X_b$;
		\item $\bigcup_{a \in A} X_a = X$;
		\item $X_a$ is contractible for each $a$;
		\item $\bigcup_{a<b} X_a \subset X_b$ is a closed cofibration.
	\end{itemize}
\end{definition}

\begin{proposition}
	If a topological space $X$ admits a \textit{cellular $A$-decomposition}, then there is a homotopy equivalence between $X$ and the geometric realization of $A$.
	\todo{@andrea,@paolo: We need a reference for this proposition. Is this Quillen's theorem A? I am also happy with saying something like:  well know result of McCord ’66 and Quillen ’73 ... Also, I think this statement should be of the form: then the natural map ??? is a homotopy equivalence.}
\end{proposition}

The case of interest for us is the following.

\begin{proposition}[{\cite[\S1.13]{berger1997confspacemodel}}]\label{p:berger}
	If a space has a cellular $\CG_d(r)$-decomposition then it is homotopy equivalent to
	the configuration space
	$\con{r}(\R^{d})$.
\end{proposition}

We warn the reader that configuration spaces themselves do not admit a cellular decomposition with respect to graph complete posets, but they do so with respect to some extended versions of these posets \cite{beuckelmann2021master}.
\begin{equation}\label{eq:extended complete graph map}
	...
\end{equation}
\todo{@andrea,@paolo: We therefore have natural homotopy equivalences ??? for each r and d.}