\subsection{Recognition of configuration spaces}\label{ss:recognition}
\todo{@paolo: please write this subsection.}

\anibal{
	This subsection needs to be rewritten.
	It should say what a complete graph filtration is.
	It should cite as a proposition that any space with one is equipped with an explicit (in terms of the filtration) weak-equivalence to the realization of the complete graph poset.
	It should also hint or describe the complete graph filtration on configuration space.
	The presentation here and the examples for the Surjection and Barratt--Eccles complexes should be made compatible with each other, then moved to their respective subsections.
	Please use $\con(r,d)$ for the configurations space of $r$-points in $\R^d$.
}

Let $F_k(\R^{d})$ denote the configuration space of $k$-tuples of pairwise disjoint vectors in $\R^{d}$.
This space is equipped with a free action of the symmetric group $\sym_k$ of permutations of $\{1,\dots,k\}$ swapping elements of a $k$-tuple.

%In this section we refine the filtrations of McClure-Smith and Barratt--Eccles in order to show that $tc$ induces equivariant weak equivalence on the filtation terms.

\begin{definition}[Complete graphs]
	A \textit{complete graph} on $k$ vertices is
	a pair $(\mu,\sigma)$, where $\mu$ is a collection of non-negative integers $\mu_{ij}\in\mathbb{N}$, $1 \leq i < j \leq k$
	and $\sigma$ is an ordering of
	$\{1,\dots,k\}$.
	We can draw a graph with $k$ vertices and with an oriented edge between any two vertices $i,j$
	labeled by $\mu_{ij}$, such that the orientation of the edges respect the ordering.
	For example the figure represents a complete graph on 4 vertices with ordering $\sigma=(1432)$
	\begin{equation*}
		\begin{tikzcd}
			\ & 2 & \  \\
			1 \arrow[ur,"2"] \arrow[rr] \arrow[dr,"3"']& \arrow[r,"1"] \arrow[u,"2"] & 3 \arrow[ul,"1"']  \\
			\ & 4 \arrow[ur,"4"'] \arrow[uu] & \
		\end{tikzcd}
	\end{equation*}
	Let us denote the set of complete graphs with $k$ vertices as $\mathcal{CG}(k)$.

	We write $\sigma_{ij}$ for the restriction of the ordering $\sigma$ to the set $\{i,j\}$.
\end{definition}

\begin{definition}[Complete graphs form a poset]
	$\mathcal{CG}(k)$ has a poset structure as follow:
	\begin{equation*}
		(\mu,\sigma)\le (\nu,\tau) \ \ \text{ if and only if } \ \ (\mu_{ij}<\nu_{ij}) \ \ \text{ or } \ \ (\mu_{ij},\sigma_{ij})= (\nu_{ij},\tau_{ij})
	\end{equation*}
	for each pair $\left\lbrace i,j\right\rbrace \subset\left\lbrace 1,\dots,k  \right\rbrace $.
\end{definition}

\begin{definition}[Filtration of the surjection multisimplicial set $Sur$]
	Fix a surjection $f\in \sur(k)_{i_{1},\dots,i_{k}}$.
	For any pair $(i,j)$ with $i< j$, let $f_{ij}$ be the subsequence of $f(1) \dots f(i_1+\dots+i_k+k)$ obtained omitting all the occurrences of elements different from $i$ and $j$.
	%		For example, let $31231 \in \sur(3)_{2,1,2}$, we have $(31231)_{12}=121$, $(31231)_{23}=323$ and $(31231)_{13}=3131$.
	The surjection $f$ belongs to $\sur_{(\mu,\sigma)}(k)\subseteq \sur(k)$ if for any pair $(i,j)$ with $i< j$, either in the sequence $f_{ij}$ $i$ and $j$ alternate strictly less than $\mu_{ij}$ times,
	%		, that is the amount of times that the order of $i$ and $j$ change ,
	or $i$ and $j$ alternate in the sequence $f_{ij}$  exactly $\mu_{ij}$ times and the ordering formed by the first occurrences of $i$ and $j$ in $f_{ij}$ agrees with $\sigma_{ij}$.
	Notice that
	\begin{equation*}
		\sur_{n}(k)=\bigcup_{\max_{i<j}(\mu_{ij})< n} \sur_{(\mu,\sigma)}(k)
	\end{equation*}
	%		\\
	%		\\
	%		Returning to the example above we have $\mu_{12}(31231)=1$, $\mu_{23}(31231)=1$ and $\mu_{13}(31231)=2$ so that $31231\in \sur_{(\mu,\sigma)}(3)_{2,1,2}$
	%		\begin{itemize}
		%		\item if $\mu_{12}>1$, $\mu_{23}>1$ and $\mu_{13}>2$ and any $\sigma\in \Sigma_{3}$;
		%		\item if some $\mu_{ij}=\mu(f)_{ij}$ and the first occurences of $i$ and $j$ in $f_{ij}$ agrees with $\sigma_{ij}$.
		%		\end{itemize}
	%	    In our example we obtain that, by equation \ref{def},
	%	    $31231\in \sur_{3}(k)_{2,1,2}$
	%	    \\
	%	    \\
	% We have obtained the filtration $$\sur_1(k) \subset \sur_2(k)  \subset \sur_3(k)  \subset \dots $$
	% that induces the filtration of the correspondent diagonal simplici%al set
	%	    $$\sur_1(k)^{D} \subset \sur_2(k)^{D}  \subset \sur_3(k)^{D}  \subset\dots $$
\end{definition}

\begin{definition}[Filtration of the Barratt--Eccles simplicial set $\cW\sym_k$]
	Fix an element $w\in (\cW\sym_k)_{d}$, $w=(w_{1},\dots , w_{d})$ with $w_{h}\in \sym_k$.
	For any pair $(i,j)$ with $i< j$, denote $w_{ij}=(w_{1,ij},\dots , w_{d,ij})$ where $w_{h,ij}$ is the subsequence of $w_{h}$ obtained omitting all the occurrences of elements different from $i$ and $j$.
	%	For example, let $(123,231,312) \in (\cW\sym_3)_{3}$, we have $(123,231,312)_{12}=(12,21,12)$, $(123,231,312)_{23}=(23,23,32)$ and $(123,231,312)_{13}=(13,31,31 )$.
	The element $w$ belongs to $\cW_{(\mu,\sigma)}\sym_k\subseteq \cW\sym_k$ if for all pairs $(i,j)$, $i< j$ either along the sequence $w_{ij}$ $i$ and $j$ swap less than $\mu_{ij}$ times,
	%	, that is, as before, the amount of times that the order of $i$ and $j$ change ,
	or they swap exactly $\mu_{ij}$ times and the first permutation $w_{1,ij}$ is equal to $\sigma_{ij}$.
	Notice that
	\begin{equation*}
		\label{def}
		\cW_{n}\sym_k=\bigcup_{\max_{i<j} (\mu_{ij})< n} \cW_{(\mu,\sigma)}\sym_{k}
	\end{equation*}
	\\
	\\
	%	Returning to the example above we have $\mu_{12}(123,231,312)=2$, $\mu_{23}(123,231,312)=1$ and $\mu_{13}(123,231,312)=1$ so that $(123,231,312)\in (\cW_{(\mu,\sigma)}\sym_3)_{3}$
	%	\begin{itemize}
		%		\item if $\mu_{12}>2$, $\mu_{23}>1$ and $\mu_{13}>1$ and any $\sigma\in \sym_{3}$;
		%		\item if some $\mu_{ij}=\mu_{ij}(w)$ and the first permutation of $w$ is equal to $\sigma_{ij}$.
		%	\end{itemize}
	%	In our example we obtain that, by equation \ref{def},
	%	$(123,231,312)\in (\cW_{3}\sym_3)_{3}$
	%	\\
	%	\\
	%	We have obtained the filtration $$\cW_1\sym_k \subset \cW_2 \sym_k
	%	\subset \cW_3 \sym_k
	%	\subset \dots $$
	%	that is the other one we need in $tc$ definition.
\end{definition}
