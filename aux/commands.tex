% !TEX root = ../msimplicial.tex

\newcommand{\mSet}{\mathsf{mSet}}
\newcommand{\stspx}[1]{\simplex^{\!#1}}
\newcommand{\msimplex}[1]{\simplex^{\!\times\mathit{#1}}}
\newcommand{\dsimplex}{\simplex^{n_1} \times \dots \times \simplex^{n_k}}
\newcommand{\diag}{\rD}

\DeclareMathOperator{\cs}{\mathfrak{cs}}
\DeclareMathOperator{\ez}{\mathfrak{ez}}
\DeclareMathOperator{\CS}{CS}

\DeclareMathOperator{\schainsUM}{N^{\simplex}_{\UM}}
\DeclareMathOperator{\mchainsk}{\chains^{\msimplex{k}}}
\DeclareMathOperator{\mchainskUM}{\chains^{\msimplex{k}}_{\UM}}
\DeclareMathOperator{\copr}{\Delta}
\DeclareMathOperator{\aug}{\epsilon}
\DeclareMathOperator{\pr}{\ast}
\DeclareMathOperator{\sh}{\mathfrak{sh}}

\newcommand{\paolo}[1]{\textcolor{green}{\underline{Paolo}: #1}}
\newcommand{\andrea}[1]{\textcolor{yellow}{\underline{Andrea}: #1}}